\chapter{Recommandations opérationnelles}
\label{app:recommandations}

\section{Framework d'évaluation ROI IA}

Cette section propose une grille d'analyse pratique en 5 dimensions permettant aux dirigeants de PME-ETI d'évaluer objectivement la pertinence d'un investissement IA.

\subsection{Matrice d'évaluation multidimensionnelle}

\begin{longtable}{@{}p{3cm}p{2cm}p{4cm}p{5cm}@{}}
\toprule
\textbf{Dimension} & \textbf{Poids} & \textbf{Critères d'évaluation} & \textbf{Méthode de scoring (1-5)} \\
\midrule
Impact business & 30\% & Gains de productivité, différenciation, chiffre d'affaires & 1=Marginal, 5=Transformationnel \\
Faisabilité technique & 25\% & Complexité, infrastructure, compétences & 1=Très complexe, 5=Simple \\
Adoption organisationnelle & 20\% & Résistances, formation, change management & 1=Forte résistance, 5=Adoption facile \\
Investissement requis & 15\% & Budget, temps, ressources humaines & 1=Très élevé, 5=Faible \\
Risques & 10\% & Technique, réglementaire, réputation & 1=Risques élevés, 5=Risques faibles \\
\bottomrule
\end{longtable}

\textbf{Calcul du score composite :}
\[
\text{Score} = \sum_{i=1}^{5} \text{Poids}_i \times \text{Score}_i
\]

\textbf{Grille de décision :}
\begin{itemize}
    \item Score $\geq$ 4,0 : Go immédiat, priorité haute
    \item Score 3,0-3,9 : Go conditionnel, pilote recommandé
    \item Score 2,0-2,9 : Attendre, améliorer les conditions
    \item Score $<$ 2,0 : No-go, revoir la stratégie
\end{itemize}

\subsection{Cas d'usage typiques et scoring}

\begin{longtable}{@{}p{4cm}p{1.5cm}p{1.5cm}p{1.5cm}p{1.5cm}p{1.5cm}p{1.5cm}@{}}
\toprule
\textbf{Cas d'usage} & \textbf{Impact} & \textbf{Faisab.} & \textbf{Adopt.} & \textbf{Invest.} & \textbf{Risque} & \textbf{Score} \\
\midrule
Rédaction assistée & 3 & 5 & 4 & 5 & 5 & 4,1 \\
Traitement documents & 4 & 4 & 4 & 4 & 4 & 4,0 \\
Chatbot client & 4 & 3 & 3 & 3 & 3 & 3,3 \\
Analyse prédictive & 5 & 2 & 2 & 2 & 2 & 2,7 \\
Automatisation RH & 4 & 3 & 2 & 3 & 2 & 2,9 \\
\bottomrule
\end{longtable}

\subsection{Outils de calcul ROI détaillés}

\textbf{Template de calcul ROI (Excel/Google Sheets) :}

\begin{longtable}{@{}p{4cm}p{3cm}p{3cm}p{4cm}@{}}
\toprule
\textbf{Catégorie} & \textbf{Variable} & \textbf{Formule} & \textbf{Exemple PME 50 ETP} \\
\midrule
Gains mensuels & Heures économisées & H/sem × 4,3 × coût horaire × adoption & 2 h × 4,3 × 45 € × 60\% = 232 € \\
Gains qualitatifs & Amélioration qualité & Score subjectif × impact CA & 20\% amélioration × 10 k€ = 2 k€ \\
Coûts formation & Formation équipes & Nb jours × tarif × participants & 2 j × 1750 € × 1 = 3500 € \\
Coûts conseil & Accompagnement & Nb jours × tarif consultant & 3 j × 400 € = 1200 € \\
Coûts licences & Outils IA & Nb utilisateurs × coût mensuel & 50 × 20 € = 1000 €/mois \\
\bottomrule
\end{longtable}

\textbf{Indicateurs de suivi post-implémentation :}
\begin{itemize}
    \item Taux d'adoption réel vs prévu
    \item Gains de productivité mesurés
    \item Satisfaction des utilisateurs (NPS interne)
    \item Incidents et temps de résolution
    \item Évolution de la qualité des livrables
\end{itemize}

\section{Checklist sélection prestataire}

Cette section propose une grille d'évaluation pondérée pour choisir un accompagnateur IA adapté aux spécificités des PME-ETI.

\subsection{Critères d'évaluation pondérés}

\begin{longtable}{@{}p{4cm}p{2cm}p{8cm}@{}}
\toprule
\textbf{Critère} & \textbf{Poids} & \textbf{Questions d'évaluation} \\
\midrule
Expérience sectorielle & 25\% & A-t-il des références dans votre secteur ? Comprend-il vos enjeux métier spécifiques ? \\
Approche pédagogique & 20\% & Propose-t-il de la formation ? Adapte-t-il son approche aux résistances ? \\
Capacité de delivery & 20\% & Peut-il implémenter concrètement ? A-t-il des ressources techniques ? \\
Références clients & 15\% & Peut-il fournir des témoignages PME-ETI ? Les résultats sont-ils documentés ? \\
Méthodologie & 10\% & A-t-il un processus structuré ? Propose-t-il des livrables clairs ? \\
Tarification & 10\% & Les tarifs sont-ils transparents ? Le modèle est-il adapté aux PME-ETI ? \\
\bottomrule
\end{longtable}

\subsection{Grille de notation détaillée}

\textbf{Expérience sectorielle (25\%)}
\begin{itemize}
    \item 5 : 3+ références sectorielles, expertise métier démontrée
    \item 4 : 2 références sectorielles, bonne compréhension des enjeux
    \item 3 : 1 référence sectorielle, compréhension générale
    \item 2 : Pas de référence sectorielle mais expérience connexe
    \item 1 : Aucune expérience sectorielle pertinente
\end{itemize}

\textbf{Approche pédagogique (20\%)}
\begin{itemize}
    \item 5 : Formation intégrée, méthodologie de change management éprouvée
    \item 4 : Formation proposée, approche structurée
    \item 3 : Formation basique, peu d'accompagnement au changement
    \item 2 : Formation optionnelle, approche technique
    \item 1 : Pas de formation, approche purement technique
\end{itemize}

\textbf{Capacité de delivery (20\%)}
\begin{itemize}
    \item 5 : Équipe technique interne, implémentation end-to-end
    \item 4 : Partenaires techniques fiables, coordination assurée
    \item 3 : Réseau de freelances, coordination variable
    \item 2 : Sous-traitance externe, peu de contrôle
    \item 1 : Pas de capacité d'implémentation
\end{itemize}

\subsection{Questions types à poser aux prestataires}

\textbf{Questions de qualification initiale :}
\begin{enumerate}
    \item « Pouvez-vous nous présenter 3 cas clients similaires avec résultats chiffrés ? »
    \item « Comment gérez-vous les résistances au changement dans nos équipes ? »
    \item « Quel est votre processus de passage du pilote au déploiement ? »
    \item « Comment assurez-vous la conformité RGPD de vos solutions ? »
    \item « Proposez-vous un support post-implémentation ? »
\end{enumerate}

\textbf{Questions d'approfondissement :}
\begin{enumerate}
    \item « Comment mesurez-vous le ROI de vos interventions ? »
    \item « Quelle est votre approche si les objectifs ne sont pas atteints ? »
    \item « Disposez-vous de certifications ou labels qualité ? »
    \item « Comment gérez-vous la montée en compétences de nos équipes ? »
    \item « Quels sont vos partenaires technologiques privilégiés ? »
\end{enumerate}

\section{Templates et outils pratiques}

Cette section fournit des ressources opérationnelles directement utilisables par les dirigeants de PME-ETI et les entrepreneurs du secteur.

\subsection{Modèle de cahier des charges IA}

\textbf{Structure recommandée :}

\begin{longtable}{@{}p{3cm}p{11cm}@{}}
\toprule
\textbf{Section} & \textbf{Contenu détaillé} \\
\midrule
Contexte entreprise & Secteur, taille, enjeux business, maturité digitale, contraintes \\
Objectifs projet & Objectifs quantifiés, délais, budget, critères de succès \\
Cas d'usage ciblés & Description détaillée, volumétrie, fréquence, acteurs impliqués \\
Contraintes techniques & SI existant, données disponibles, sécurité, conformité RGPD \\
Livrables attendus & Formation, documentation, outils, support, transfert de compétences \\
Modalités projet & Organisation, planning, jalons, comité de pilotage \\
Critères de sélection & Expérience, références, méthodologie, tarification \\
\bottomrule
\end{longtable}

\subsection{Grille d'audit IA interne}

\textbf{Diagnostic préalable (auto-évaluation) :}

\begin{longtable}{@{}p{4cm}p{2cm}p{2cm}p{6cm}@{}}
\toprule
\textbf{Dimension} & \textbf{Note /5} & \textbf{Poids} & \textbf{Plan d'action si < 3} \\
\midrule
Maturité des données & \_\_/5 & 25\% & Audit qualité des données, gouvernance, nettoyage \\
Compétences internes & \_\_/5 & 20\% & Formation, recrutement, sensibilisation \\
Infrastructure IT & \_\_/5 & 20\% & Modernisation, cloud, sécurité \\
Culture innovation & \_\_/5 & 15\% & Change management, communication \\
Budget disponible & \_\_/5 & 10\% & Business case, recherche de financement \\
Support direction & \_\_/5 & 10\% & Sensibilisation CODIR, sponsor \\
\bottomrule
\end{longtable}

\textbf{Score de maturité IA :}
\[
\text{Maturité} = \sum \text{Note} \times \text{Poids}
\]

\textbf{Recommandations par niveau :}
\begin{itemize}
    \item 4,0-5,0 : Prêt pour déploiement ambitieux
    \item 3,0-3,9 : Prêt pour pilote structuré
    \item 2,0-2,9 : Préparation nécessaire (6-12 mois)
    \item < 2,0 : Fondamentaux à construire (12-18 mois)
\end{itemize}

\subsection{Indicateurs de suivi projet}

\textbf{Dashboard de pilotage (hebdomadaire) :}

\begin{longtable}{@{}p{4cm}p{3cm}p{3cm}p{4cm}@{}}
\toprule
\textbf{KPI} & \textbf{Cible} & \textbf{Réalisé} & \textbf{Action si écart} \\
\midrule
Avancement planning & 100\% & \_\_\% & Réajustement des ressources/scope \\
Participation formation & 90\% & \_\_\% & Communication, motivation \\
Adoption outils & 60\% & \_\_\% & Support utilisateur, formation \\
Incidents techniques & < 2/sem & \_\_/sem & Support technique, debug \\
Satisfaction équipe & > 4/5 & \_\_/5 & Amélioration UX, formation \\
\bottomrule
\end{longtable}

\textbf{Reporting mensuel dirigeant :}
\begin{itemize}
    \item Synthèse de l'avancement vs planning initial
    \item Gains de productivité mesurés (heures, qualité)
    \item Budget consommé vs prévu
    \item Risques identifiés et plans de mitigation
    \item Recommandations pour la suite
\end{itemize}

\subsection{Bonnes pratiques organisationnelles}

\textbf{Gouvernance projet IA :}
\begin{enumerate}
    \item \textbf{Sponsor exécutif} : DG ou membre CODIR, garant des objectifs business
    \item \textbf{Chef de projet métier} : Responsable opérationnel, interface quotidienne
    \item \textbf{Référent technique} : DSI ou expert IT, garant de l'architecture et de la sécurité
    \item \textbf{Champions utilisateurs} : 2-3 early adopters par service concerné
\end{enumerate}

\textbf{Rituels projet recommandés :}
\begin{itemize}
    \item \textbf{Daily stand-up} (phase pilote) : Point quotidien de l'équipe projet
    \item \textbf{Weekly review} : Avancement, blocages, décisions
    \item \textbf{Monthly steering} : Reporting dirigeant, arbitrages stratégiques
    \item \textbf{Quarterly business review} : ROI, évolutions, roadmap
\end{itemize}

\textbf{Communication et conduite du changement :}
\begin{itemize}
    \item Kick-off général : présentation de la vision, des bénéfices, du planning
    \item Newsletter projet : actualités, témoignages, bonnes pratiques
    \item Sessions Q\&A : réponses aux interrogations, démystification
    \item Célébration des succès : reconnaissance des early adopters, partage des résultats
\end{itemize}
