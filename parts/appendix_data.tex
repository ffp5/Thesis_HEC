\chapter{Données primaires}
\label{app:data}

Cette annexe présente des données anonymisées issues des 63 entretiens et des 5 propositions commerciales analysées. Elle complète la méthodologie (Annexe \ref{app:methodologie}) et alimente les analyses du chapitre \ref{chap:field_diagnosis} ainsi que les recommandations du chapitre \ref{chap:recommendations}.

\section{Échantillon des contacts prospectés (anonymisé)}
\subsection{Répartition par secteur}
\begin{longtable}{@{}p{6cm}p{3cm}p{3cm}p{3cm}@{}}
\toprule
\textbf{Secteur} & \textbf{Part (\%)} & \textbf{Taille médiane (ETP)} & \textbf{Rendez-vous obtenus} \\
\midrule
Conseil et services & 32 & 80 & 5 \\
Industrie (manufacturier, distribution spécialisée) & 25 & 120 & 3 \\
Services B2B (marketing, formation, communication) & 21 & 65 & 3 \\
Technologie/Digital (éditeurs, agences) & 15 & 70 & 1 \\
Finance/Assurance (banques régionales, mutuelles) & 7 & 150 & 1 \\
\midrule
\textbf{Total} & \textbf{100} & \textbf{—} & \textbf{13} \\
\bottomrule
\end{longtable}
Note : 63 contacts initiaux, 13 rendez-vous obtenus (taux de 20,6\%).

\subsection{Répartition par rôle des interlocuteurs}
\begin{longtable}{@{}>{\raggedright\arraybackslash}p{6cm}>{\raggedright\arraybackslash}p{2.5cm}>{\raggedright\arraybackslash}p{5cm}@{}}
\toprule
\textbf{Rôle} & \textbf{Part (\%)} & \textbf{Observations principales} \\
\midrule
Direction générale (DG/CEO) & 38 & Décision ROI/risque, sponsor potentiel \\
Managers opérationnels (COO/Directeur BU) & 34 & Priorisation des cas d'usage, charge opérationnelle \\
IT/Data (DSI/RSI/Data Lead) & 28 & Sécurité, RGPD, maintenabilité \\
\bottomrule
\end{longtable}

\subsection{Agrégats d'entonnoir (mois type)}
\begin{longtable}{@{}>{\raggedright\arraybackslash}p{5.5cm}>{\raggedright\arraybackslash}p{2.5cm}>{\raggedright\arraybackslash}p{2.5cm}>{\raggedright\arraybackslash}p{3cm}@{}}
\toprule
\textbf{Étape} & \textbf{Volume} & \textbf{Conversion} & \textbf{Délai médian} \\
\midrule
Prospects contactés (cold + social) & 120 & — & — \\
Rendez-vous obtenus & 25 & 20,8\% & 7 jours \\
Rendez-vous qualifiés (BANT) & 15 & 60\% & 10 jours \\
Propositions émises & 10 & 66\% & 5 jours \\
Contrats signés & 6 & 60\% & 4–8 semaines (cycle) \\
\bottomrule
\end{longtable}

\section{Extraits d'entretiens clés (anonymisés)}
\begin{itemize}
    \item E12 (DG, services B2B) : « Si vous me montrez un retour sur investissement en 3 mois, on lance. »
    \item E27 (COO, industrie) : « Sans suivi, la formation n'a pas changé nos processus. »
    \item E39 (DRH, conseil) : « Le sujet n'est pas la technologie, c'est embarquer les managers. »
    \item E18 (DSI, PME) : « La complexité perçue nous freine plus que le budget. »
    \item E31 (Directeur BU, services) : « Quatre validations pour un pilote d'un mois... »
    \item E07 (DG, industrie) : « Qui pilote l'IA chez nous ? Personne clairement. »
\end{itemize}

\section{Propositions commerciales — détails agrégés}
\begin{longtable}{@{}>{\raggedright\arraybackslash}p{3.2cm}>{\raggedright\arraybackslash}p{5cm}>{\raggedright\arraybackslash}p{2.5cm}>{\raggedright\arraybackslash}p{4.5cm}@{}}
\toprule
\textbf{Client (anonymisé)} & \textbf{Objet de l'intervention} & \textbf{Montant (\texteuro{} HT)} & \textbf{Résultats (12–18 mois)} \\
\midrule
Aesio (communication) & Formation + optimisation Copilot + ateliers & 3\,200 & Délai de production : 65 jours $\rightarrow$ 18 jours (-72\%), +35\% de productivité \\
Antilogy (conseil) & Programme de formation (15 collaborateurs) + cadrage & 3\,500 & Adoption de 70\% de l'équipe cible, 2 cas d'usage déployés \\
Intégrhale (recrutement) & Formation + automatisations sourcing/formatage & 2\,600 & Sourcing -40\%, 2 heures/semaine libérées par consultant \\
Carecall (santé B2B) & Génération de leads automatisée (MVA) & 2\,500 & +28\% de leads qualifiés, coût/opportunité -22\% \\
Tectona (PME mobilier) & Formation managériale + audit vertical & 3\,500 & Backlog priorisé, pilote documentaire lancé \\
\bottomrule
\end{longtable}
Notes : montants issus des propositions \cite{luwai2025aesio, luwai2025antilogy, luwai2025integrhale, luwai2025carecall, luwai2025tectona}. Résultats mesurés ou déclarés selon les cas.

\section{Mesures et KPIs de suivi (pilotes)}
\begin{longtable}{@{}p{5.5cm}p{7.5cm}p{4cm}@{}}
\toprule
\textbf{Indicateur} & \textbf{Définition/Méthode} & \textbf{Cible 12 semaines} \\
\midrule
Adoption effective & Pourcentage d'utilisateurs actifs 1 fois/jour ouvré & $\geq 60\%$ \\
Gain de productivité & Heures gagnées par personne (baseline vs fin pilote) & +20–30\% \\
Délai de première valeur & Jours entre le début et le premier livrable utile & $\leq 28$ jours \\
Qualité perçue & Score de 1 à 5 sur les outputs IA (panel interne) & $\geq 4{,}0$ \\
Conformité & Incidents RGPD ; complétude du registre & 0 incident ; 100\% \\
\bottomrule
\end{longtable}

\section{Cadre de calcul ROI — rappel opérationnel}
Rappel du cadre présenté en section \ref{sec:roi_framework} :
\begin{itemize}
    \item Gains mensuels $G = \text{heures/semaine} \times 4{,}3 \times \text{coût horaire} \times \text{taux d'adoption}$.
    \item Coûts $C = \text{formation} + \text{conseil} + \text{licences} + \text{temps interne}$.
    \item $\text{ROI}_T = \frac{T \times G - C}{C}$.
\end{itemize}
Exemple pour une PME de 40 ETP (services) — voir section \ref{sec:roi_framework}.
