\section*{Résumé en français}

Cette thèse analyse les paradoxes de l'adoption de l'intelligence artificielle dans les PME-ETI françaises à travers le prisme entrepreneurial du cas Luwai. Malgré un écosystème d'innovation reconnu et des avancées réglementaires pionnières, la France présente un écart significatif entre le potentiel théorique de l'IA et son adoption effective dans le tissu économique.

L'étude s'appuie sur 63 entretiens prospects, 5 propositions commerciales réelles et 9 mois d'expérience entrepreneuriale pour identifier les résistances spécifiques aux entreprises françaises. Les principales barrières identifiées incluent : (1) une méconnaissance des cas d'usage concrets de l'IA, (2) des résistances culturelles liées à la peur du changement organisationnel, (3) un manque de compétences internes pour évaluer et implémenter ces technologies, et (4) des contraintes budgétaires et temporelles pour des projets perçus comme expérimentaux.

La recherche propose un modèle entrepreneurial "Formation-Conseil-Delivery" testé via Luwai, permettant de transformer ces résistances en opportunités de création de valeur. Les résultats montrent des gains de productivité documentés de 20 à 40\% sur les tâches automatisées, avec un ROI moyen de 300\% sur 12 mois pour les clients accompagnés.

Cette thèse contribue à la littérature sur l'adoption technologique en proposant une taxonomie opérationnelle des freins à l'IA spécifiques au contexte français, ainsi qu'un framework pratique pour les entrepreneurs et dirigeants du secteur.

\textbf{Mots-clés :} Intelligence artificielle, adoption technologique, entrepreneurship, PME-ETI, transformation numérique, résistances organisationnelles, modèles d'affaires

\section*{Executive Summary (English)}

This thesis analyzes the paradoxes of artificial intelligence adoption in French SMEs through the entrepreneurial lens of the Luwai case study. Despite a recognized innovation ecosystem and pioneering regulatory advances, France shows a significant gap between AI's theoretical potential and its effective adoption in the economic fabric.

The study draws on 63 prospect interviews, 5 real commercial proposals, and 9 months of entrepreneurial experience to identify specific resistances in French companies. Main barriers identified include: (1) lack of knowledge about concrete AI use cases, (2) cultural resistance related to fear of organizational change, (3) lack of internal skills to evaluate and implement these technologies, and (4) budget and time constraints for projects perceived as experimental.

The research proposes a "Training-Consulting-Delivery" entrepreneurial model tested through Luwai, enabling the transformation of these resistances into value creation opportunities. Results show documented productivity gains of 20-40\% on automated tasks, with an average ROI of 300\% over 12 months for supported clients.

This thesis contributes to the technology adoption literature by proposing an operational taxonomy of AI barriers specific to the French context, as well as a practical framework for entrepreneurs and sector leaders.

\textbf{Keywords:} Artificial intelligence, technology adoption, entrepreneurship, SMEs, digital transformation, organizational resistance, business models
