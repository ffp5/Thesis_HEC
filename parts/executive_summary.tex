\section*{Résumé en français}

Ce travail de thèse s’intéresse aux paradoxes de l’adoption de l’intelligence artificielle par les petites et moyennes entreprises (PME) et les entreprises de taille intermédiaire (ETI) françaises sous l’angle entrepreneurial, à partir d’une observation approfondie du cas Luwai. L’IA est un enjeu d’actualité aux quatre coins du monde, et la France est souvent saluée pour la qualité de son écosystème de start-ups, de ses politiques publiques innovantes ou encore de sa régulation pionnière en la matière. Toutefois, des études et rapports internationaux relèvent que la France se place généralement en queue de peloton des pays développés en matière d’adoption et d’investissement en IA par ses entreprises.
\\\\
Ce travail s’appuie sur plus de 500 appels de prospection, 63 entretiens avec des entreprises différentes, 5 propositions commerciales concrètes et près de trois mois d’expérience d’entrepreneur à la tête de Luwai pour comprendre les résistances situées au niveau de l’entreprise. Parmi celles-ci, on retrouve :
(1) une connaissance limitée de l’IA et de ses cas d’usage,
(2) une résistance culturelle au changement de la part des employés et des managers,
(3) un manque de compétences internes pour évaluer une solution, l’implémenter et la délivrer,
(4) une priorité budgétaire et temporelle peu favorable à de tels projets, perçus comme expérimentaux et non cruciaux.
\\\\
Cette recherche propose un modèle entrepreneurial « Formation-Conseil-Delivery » que nous avons voulu tester *in vivo* via l’expérience entrepreneuriale chez Luwai. Le but est que ce modèle puisse permettre de transformer les résistances en opportunités de création de valeur pour les clients. Les bénéfices clients sont notamment documentés par des gains de productivité de 20 \% à 40 \% sur les tâches automatisées, avec un ROI élevé de 300 \% sur 12 mois pour les clients accompagnés.
\\\\
Cette contribution à la littérature sur l’adoption de technologie propose une taxonomie opérationnelle des freins à l’IA dans le contexte français, ainsi qu’un modèle pratique pour les entrepreneurs et chefs d’entreprises du secteur.
\\\\
\textbf{Mots-clés} : intelligence artificielle, adoption technologique, entrepreneurship, PME-ETI, transformation numérique, résistances organisationnelles, modèles d’affaires


\newpage
\section*{Résumé en anglais (Executive Summary)}

This thesis analyzes the paradoxes of artificial intelligence adoption in French SMEs through the entrepreneurial lens of the Luwai case study. Despite a recognized innovation ecosystem and pioneering regulatory advances, France shows a significant gap between AI’s theoretical potential and its effective adoption in the economic fabric.
\\\\
The study draws on 63 prospect interviews, 5 real commercial proposals, and three months of entrepreneurial experience to identify specific resistances in French companies. Main barriers identified include:
(1) lack of knowledge about concrete AI use cases,
(2) cultural resistance related to fear of organizational change,
(3) lack of internal skills to evaluate and implement these technologies, and
(4) budget and time constraints for projects perceived as experimental.
\\\\
The research proposes a “Training-Consulting-Delivery” entrepreneurial model tested through Luwai, enabling the transformation of these resistances into value creation opportunities. Results show documented productivity gains of 20-40 \% on automated tasks, with an expected average ROI of 300 \% over 12 months for supported clients.
\\\\
This thesis contributes to the technology adoption literature by proposing an operational taxonomy of AI barriers specific to the French context, as well as a practical framework for entrepreneurs and sector leaders.
\\\\
\textbf{Keywords} : artificial intelligence, technology adoption, entrepreneurship, SMEs, digital transformation, organizational resistance, business models
