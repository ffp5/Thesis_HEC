\chapter{Méthodologie de recherche}
\label{app:methodologie}

Cette annexe détaille le dispositif méthodologique ayant permis de produire le diagnostic terrain (chapitre \ref{chap:field_diagnosis}) et d'alimenter les recommandations (chapitre \ref{chap:recommendations}). Elle précise le design de recherche, les critères d'échantillonnage, le protocole d'entretien, la démarche d'analyse, les mesures de validité et de fiabilité ainsi que les limites.

\section{Design de recherche et posture}
La recherche adopte un design qualitatif interprétatif, avec une posture d'observation participante du fondateur de Luwai. L'objectif est d'identifier des mécanismes récurrents (résistances, leviers) et de proposer des cadres décisionnels actionnables. Le dispositif combine :
\begin{itemize}
    \item \textbf{Entretiens semi-directifs} (n=63) auprès de dirigeants et managers en PME-ETI.
    \item \textbf{Analyse de propositions commerciales} (n=5) en tant que traces matérielles des interactions.
    \item \textbf{Observations} des cycles de vente, des ateliers et des pilotes (\emph{Minimum Viable Automation}).
\end{itemize}

\section{Échantillonnage et critères d'inclusion}
\subsection{Cadre de sélection}
\begin{itemize}
    \item \textbf{Taille}: 50 à 500 ETP (PME-ETI).
    \item \textbf{Localisation} : France métropolitaine (avec une sur-représentation de l'Île-de-France).
    \item \textbf{Secteurs} : services B2B, conseil, industrie légère, technologie/digital, finance/assurance.
    \item \textbf{Profils} : DG/CEO, COO, DRH, directeur BU, responsable IT/Data.
    \item \textbf{Maturité IA} : variable, allant de \emph{AI-curious} à \emph{AI-pilots}.
\end{itemize}

\subsection{Répartition agrégée (synthèse)}
Voir l'annexe Données (\ref{app:data}) pour le détail chiffré. En résumé :
\begin{itemize}
    \item Conseil et services (32\%), Industrie (25\%), Services B2B (21\%), Technologie/Digital (15\%), Finance/Assurance (7\%).
    \item Rôles : Direction générale (38\%), Managers opérationnels (34\%), IT/Data (28\%).
\end{itemize}

\section{Protocole d'entretien semi-directif}
\subsection{Guide d'entretien (30–45 min)}
\begin{enumerate}
    \item \textbf{État des lieux IA}: outils utilisés, cas d'usage actuels, perception de la valeur.
    \item \textbf{Freins et résistances}: techniques, organisationnels, culturels, économiques.
    \item \textbf{Opportunités et besoins} : priorités, quick wins, contraintes (RGPD, sécurité).
    \item \textbf{Décision et gouvernance} : sponsor, budget, critères de succès, prochain pas.
\end{enumerate}
Exemples de questions :
\begin{itemize}
    \item « Quelles tâches consomment le plus de temps et seraient candidates à l'automatisation ? »
    \item « Comment objectivisez-vous le ROI d'une initiative IA ? »
    \item « Qui serait le sponsor et le \emph{référent IA} dans votre organisation ? »
\end{itemize}

\subsection{Collecte et éthique}
\begin{itemize}
    \item Consentement verbal préalable, anonymisation systématique des verbatims.
    \item Aucune collecte de données personnelles sensibles ; absence de données clients finales.
    \item Stockage chiffré des notes et des tableaux de codage (support : tableur/Notion).
\end{itemize}

\section{Transcription, codage et schéma d'analyse}
\subsection{Processus de codage thématique}
Les notes d'entretien ont été codées en deux passes (ouverte puis axiale). Un codebook a été stabilisé autour de \textbf{12 catégories de résistances} et \textbf{8 opportunités} (tableaux \ref{tab:codes_resistances} et \ref{tab:codes_opportunites}).

\subsection{Codebook — résistances}
\begin{longtable}{@{}p{4.5cm}p{9cm}@{}}
\toprule
\textbf{Code} & \textbf{Définition et exemples} \\
\midrule
Pas le temps / priorités court terme & Urgences opérationnelles supplantent l'IA (« on verra l'an prochain »). \\
Cycles décisionnels longs & Multiples validations ; délais incompatibles avec un pilote rapide. \\
Absence de référent IA & Aucun ownership interne ; initiatives diffuses. \\
Anxiété technique & Crainte d'erreurs et de complexité perçue. \\
Ego / fausse maîtrise & Connaissances partielles freinant l'apprentissage collectif. \\
ROI difficile à objectiver & Gains diffus et transverses, manque d'indicateurs. \\
Infrastructure legacy (perçue) & SI jugé « trop ancien », même quand contournable par le cloud. \\
Gouvernance des données faible & Absence de politique, qualité/accès non maîtrisés. \\
Peur du remplacement & Inquiétudes sociales (peu prévalentes vs peur du changement). \\
Résistance générationnelle (modérée) & Stéréotype peu prédictif ; hétérogénéité réelle. \\
Conformité / régulation & RGPD/IA Act cités comme freins ex ante. \\
Manque de cas d'usage clairs & Difficulté à prioriser et cadrer. \\
\bottomrule
\caption{Codebook — catégories de résistances}\label{tab:codes_resistances}
\end{longtable}

\subsection{Codebook — opportunités}
\begin{longtable}{@{}p{4.5cm}p{9cm}@{}}
\toprule
\textbf{Code} & \textbf{Définition et exemples} \\
\midrule
Formation / acculturation & Création d'un langage commun ; ateliers CODIR/équipe. \\
Traitement documentaire & Gain sur CVs, contrats, rapports ; contrôles qualité. \\
Veille et synthèse & Réduction du temps hebdomadaire de veille (60–80\%). \\
Rédaction assistée & Gains de 25–40\% sur emails, offres, notes. \\
FAQ interne / connaissances & Accès rapide à la connaissance ; prompts-guides. \\
Automatisation back-office & RPA+IA sur tâches récurrentes (facturation, consolidations). \\
Agents IA métiers & Copilotes ciblés ; déploiement post-pilote. \\
Gouvernance et conformité & Processus \emph{privacy by design} ; traçabilité. \\
\bottomrule
\caption{Codebook — catégories d'opportunités}\label{tab:codes_opportunites}
\end{longtable}

\subsection{Plan d'analyse et triangulation}
\begin{itemize}
    \item \textbf{Synthèses par secteur} puis consolidation inter-secteurs.
    \item \textbf{Matrices} impact $\times$ difficulté ; \textbf{typologie} des adopteurs.
    \item \textbf{Triangulation} entre verbatims, propositions commerciales et observations.
\end{itemize}

\section{Fiabilité, validité et biais}
\subsection{Fiabilité}
\begin{itemize}
    \item \textbf{Double-codage} sur un sous-échantillon (n=12) par un second codeur externe ; \textbf{Cohen's $\kappa$ = 0,78} (bonne concordance).
    \item \textbf{Stabilisation du codebook} après itérations ; journal de décision (\emph{audit trail}).
\end{itemize}

\subsection{Validité}
\begin{itemize}
    \item \textbf{Saturation thématique} atteinte autour du 50\textsuperscript{e} entretien.
    \item \textbf{Validation des répondants} ponctuelle (membres vérifications sur 6 cas).
    \item \textbf{Triangulation} des sources (entretiens, documents, observations).
\end{itemize}

\subsection{Biais et limites}
\begin{itemize}
    \item \textbf{Biais géographique} : Île-de-France sur-représentée.
    \item \textbf{Fenêtre temporelle courte} : juin–août 2025.
    \item \textbf{Posture d'observation participante} : risques de confirmation ; atténués par double-codage et verbatims anonymisés.
\end{itemize}

\section{Lien avec le cadre ROI et les recommandations}
Le \emph{cadre ROI} proposé (section \ref{sec:roi_framework}) s'appuie sur les variables observées en entretien (heures gagnées, adoption effective, coûts internes) et alimente la \textbf{matrice de décision} et la \textbf{feuille de route 90/180 jours} (chapitre \ref{chap:recommendations}).

\section{Ressources et logiciels}
\begin{itemize}
    \item Prise de notes structurée (tableur), stockage chiffré.
    \item Tableaux de codage et matrices dans un espace de travail (Notion/Sheets).
    \item Génération de tableaux \LaTeX{} (\texttt{longtable}, \texttt{booktabs}).
\end{itemize}

\section{Accès aux données agrégées}
Les agrégats, distributions et extraits anonymisés sont présentés dans l'annexe Données (\ref{app:data}). Les identifiants d'entretien sont de la forme E01–E63 ; toute donnée sensible a été supprimée ou généralisée.
