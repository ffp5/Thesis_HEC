Je remercie tout d’abord celles et ceux sans qui ce travail n’aurait pas pu voir le jour. Sans leur soutien, leur accompagnement, leurs conseils et leur confiance, je n’aurais pas pu mener cette thèse.
\\\\
À mes cofondateurs chez \textbf{Luwai}, Miguel, pour ce que nous construisons ensemble et notre complétude. Merci en particulier à \textbf{Miguel} pour ses conseils stratégiques et sa vision globale.
\\\\
Aux décideurs de \textbf{63 entreprises} qui ont accepté de me recevoir dans le cadre d’entretiens exploratoires et ont partagé leur expérience des technologies IA. Sans vous, cette étude n’aurait pu voir le jour.
\\\\
Aux élèves de la promotion \textbf{2024 et 2025 du MS X-HEC Entrepreneurs} pour leur soutien et la richesse des échanges pendant ces deux années. Je me suis nourri des expériences et retours de chacun.
\\\\
À mes clients et partenaires  \textbf{Social Army, Antilogy, Corma, Intégrhale, Carecall, Tectona} — qui me font confiance pour les accompagner dans leur transformation et ont validé par leurs résultats la pertinence de mon approche.
\\\\
Aux mentors, aux amis et aux proches qui ont épaulé mon projet, ou bien ces deux années de patience et de confiance en mes capacités. Merci d’avoir tant cru en moi.
\\\\
À l’\textbf{École Polytechnique} et \textbf{HEC Paris} pour la qualité des enseignements, l’ouverture d’esprit sur l’entrepreneuriat et l’écosystème dans lequel nous évoluons. Sans ces deux expériences, ce travail n’apporterait pas la même ambition.
\\\\
Cette thèse est autant un \textbf{cheminement} qu’une \textbf{aventure collective}. Elle est le témoignage de la \textbf{vitalité de l’écosystème français d’innovation} quand il parvient à accorder la pertinence des recherches académiques et le sens du réalisme de l’entrepreneur.