\chapter{Introduction}
\label{chap:introduction}

\section{Contexte et enjeux}

La France est aujourd’hui confrontée à un double paradoxe sur l'IA:
\\\\
D’une part, notre pays est bien positionné sur le sujet : il peut s’appuyer sur un tissu de recherche de tout premier plan, comme le confirment l’INRIA et les chaires DeepTech de l’ANR, sur une communauté de startups florissante qui compte aujourd’hui des licornes et des pépites de la tech comme Mistral AI ou Hugging Face, et sur une place de leader en Europe concernant la régulation et la gouvernance de l’IA avec l’IA Act.
\\\\
D’autre part, l’adoption de l’IA par nos entreprises est très hétérogène. En effet, si selon le baromètre du numérique 2024, 33 \% des Français déclaraient avoir utilisé des outils d’IA générative, cette utilisation reste avant tout un usage personnel et occasionnel. Dans le monde du travail, le fossé est profond entre les grandes entreprises et les PME-ETI qui font la richesse de nos territoires.
\\\\
Et cela est d’autant plus inquiétant que les enjeux sont immenses. Selon Shanghai SKY, l’IA est susceptible d’apporter une amélioration de la productivité de 20 à 40 \% pour de très nombreuses tâches. En quête de compétitivité, nous ne pouvons nous permettre de passer à côté de ces gains de performance.
\\\\
Le \textbf{paradoxe français} de l'IA se manifeste à plusieurs niveaux :
\\
\begin{itemize}
    \item \textbf{Au niveau technologique :} Nous disposons d'un écosystème d'innovation de premier plan mais peinons à diffuser ces innovations dans le tissu économique.
    \item \textbf{Au niveau organisationnel :} Les entreprises françaises excellent dans l'innovation produit mais montrent des résistances culturelles à l'adoption de nouvelles méthodes de travail.
    \item \textbf{Au niveau entrepreneurial :} L'écosystème startup français est dynamique mais les services d'accompagnement peinent à adresser efficacement le segment des PME-ETI.
\end{itemize}
\medskip
C'est dans ce contexte que s'inscrit la création de Luwai et l'expérience entrepreneuriale qui nourrit cette thèse. En tant qu'ingénieur de grandes écoles françaises ayant vécu l'adoption naturelle de l'IA dans la Silicon Valley, puis confronté aux résistances françaises, j'ai identifié une opportunité de création de valeur dans l'accompagnement des entreprises françaises vers une utilisation productive de l'IA.

\section{Problématique centrale}

Cette thèse s'articule autour d'une question fondamentale :

\begin{quote}
\textbf{Comment expliquer l'écart entre le potentiel de l'IA et son adoption effective dans les PME-ETI françaises, et quelles stratégies entrepreneuriales permettent de transformer ces résistances en opportunités de création de valeur ?}
\end{quote}

Cette problématique centrale se décline en trois sous-questions opérationnelles :

\begin{enumerate}
    \item \textbf{Quelles sont les résistances spécifiques} à l'adoption de l'IA dans les PME-ETI françaises et comment se manifestent-elles selon les secteurs et les profils d'entreprises ?
    \item \textbf{Comment construire un modèle d'affaires viable} pour accompagner ces entreprises dans leur transformation, en naviguant entre les contraintes de scalabilité et les besoins de personnalisation ?
    \item \textbf{Quels leviers entrepreneuriaux et managériaux} permettent d'accélérer l'adoption de l'IA et de maximiser son impact opérationnel ?
\end{enumerate}
\medskip
L'angle entrepreneurial adopté dans cette thèse permet d'aborder ces questions sous un prisme résolument pratique, alimenté par l'expérience concrète de construction et de développement de Luwai.

\section{Objectifs de recherche}

Cette recherche vise quatre objectifs principaux :

\begin{enumerate}
    \item \textbf{Cartographier l'écosystème et la chaîne de valeur IA en France}\\
    Identifier les acteurs clés, analyser leurs interactions et comprendre les flux de valeur dans l'accompagnement à l'adoption de l'IA, en particulier pour les PME-ETI.
    \item \textbf{Identifier les résistances organisationnelles et culturelles spécifiques}\\
    Développer une taxonomie opérationnelle des freins à l'adoption de l'IA, en distinguant les résistances techniques, organisationnelles, culturelles et économiques propres au contexte français.
    \item \textbf{Analyser le modèle entrepreneurial Luwai comme cas d'étude}\\
    Documenter et analyser l'évolution du modèle d'affaires Luwai, de sa genèse à ses pivots stratégiques, pour en extraire des apprentissages généralisables sur l'entrepreneurship dans ce secteur.
    \item \textbf{Formuler des recommandations pour entrepreneurs et décideurs}\\
    Proposer des frameworks pratiques et des recommandations actionnables pour les entrepreneurs souhaitant se positionner sur ce marché et les dirigeants de PME-ETI engagés dans leur transformation IA.
\end{enumerate}

\section{Méthodologie}

Cette recherche adopte une \textbf{approche mixte}, combinant rigueur académique et pragmatisme entrepreneurial. Elle s'appuie sur une méthodologie en trois temps :

\begin{enumerate}
    \item \textbf{Une revue de littérature} académique et professionnelle pour ancrer la recherche dans les cadres théoriques existants (modèles d'adoption technologique comme TAM et UTAUT, innovation disruptive) et le contexte de l'écosystème français (analyses sectorielles, rapports sur l'IA).

    \item \textbf{Une étude de cas approfondie} de l'entreprise Luwai. Cette approche par observation participante, en tant que CEO-fondateur, permet de documenter de l'intérieur la genèse du projet, l'évolution de son modèle d'affaires et ses pivots stratégiques.

    \item \textbf{Une collecte de données de terrain} pour confronter le modèle aux réalités du marché.
\end{enumerate}

\subsubsection{Collecte des données primaires}
La richesse empirique de cette recherche repose sur une importante collecte de données menée entre juillet et début octobre 2025 :
\begin{itemize}
    \item \textbf{Entretiens qualitatifs} : 63 entretiens ont été menés suite à une campagne de 500 appels à froid (taux de conversion de 12,6\%). Les prospects sont issus de secteurs variés : Conseil (32\%), Industrie (25\%), Services (21\%), Tech (15\%) et Finance (7\%).
    \item \textbf{Analyse documentaire} : 5 propositions commerciales réelles ont été analysées en détail (Aesio \cite{luwai2025aesio}, Antilogy \cite{luwai2025antilogy}, Intégrhale \cite{luwai2025integrhale}, Carecall \cite{luwai2025carecall}, Tectona \cite{luwai2025tectona}).
    \item \textbf{Observation directe} : Les interactions commerciales, les cycles de vente et l'évolution des besoins clients ont été suivis sur près de trois mois d'activité.
\end{itemize}

\subsubsection{Analyse des données}
La méthode d’analyse combine une approche thématique et comparative :
\begin{itemize}
    \item Codage thématique des entretiens pour identifier les freins et leviers récurrents.
    \item Identification des patterns et des cas d’usage émergents à travers les différents secteurs.
    \item Cartographie des propositions de valeur et des modèles d'affaires testés.
\end{itemize}
\medskip
Cette approche intégrée permet de relier les observations de terrain aux cadres théoriques pour produire des conclusions actionnables.

\section{Plan et contributions attendues}

Cette thèse s'organise en cinq chapitres principaux après cette introduction :
\bigskip
\begin{itemize}
    \item \textbf{Chapitre 2} : Revue de littérature et cadre théorique - Fondements académiques de l'adoption technologique et spécificités du contexte français.
    \item \textbf{Chapitre 3} : Diagnostic terrain : résistances et opportunités - Analyse empirique des freins et leviers identifiés via les entretiens.
    \item \textbf{Chapitre 4} : Cas d'étude Luwai - Documentation du modèle entrepreneurial et de son évolution.
    \item \textbf{Chapitre 5} : Recommandations et perspectives - Frameworks pratiques et implications pour l'écosystème.
    \item \textbf{Chapitre 6} : Conclusion - Synthèse des apports, limites et réflexions finales.
\end{itemize}
\medskip
Les \textbf{contributions attendues} se situent à trois niveaux :
\bigskip
\begin{itemize}
    \item \textbf{Contribution empirique} : Première étude qualitative approfondie sur les résistances à l'IA dans les PME-ETI françaises, avec une taxonomie opérationnelle des freins et leviers d'adoption.
    \item \textbf{Contribution théorique} : Extension des modèles classiques d'adoption technologique au contexte spécifique de l'IA et développement d'un framework « Formation-Conseil-Delivery » pour les services B2B.
    \item \textbf{Contribution managériale} : Guide pratique d'évaluation des opportunités IA pour les dirigeants et recommandations stratégiques pour les entrepreneurs du secteur, avec des métriques ROI documentées et des indicateurs de performance.
\end{itemize}
\medskip
Cette approche vise à combler le gap entre la recherche académique sur l'adoption technologique et les besoins concrets des praticiens confrontés aux enjeux d'implémentation de l'IA dans leurs organisations.
