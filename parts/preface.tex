\begin{quote}
\textit{"Il y a deux types d'entreprises : celles qui s'adaptent à l'IA et celles qui disparaissent."} \\
--- Jensen Huang, CEO de NVIDIA
\end{quote}

L'intelligence artificielle représente aujourd'hui l'une des transformations technologiques les plus profondes de notre époque. Pourtant, en France, cette révolution semble avancer à deux vitesses : d'un côté, un écosystème startup dynamique et des avancées réglementaires pionnières avec l'IA Act européen ; de l'autre, des PME-ETI qui peinent à concrétiser le potentiel de ces technologies dans leur quotidien opérationnel.

Cette thèse prend racine dans un \textbf{choc culturel personnel} vécu lors d'un échange de trois mois à San Francisco. En tant qu'ingénieur polytechnicien immergé dans l'écosystème de la Silicon Valley, j'ai été témoin d'une adoption massive et naturelle de l'IA dans tous les secteurs. Appels d'offres automatisés, due diligences accélérées, créations de contenu optimisées : l'IA était omniprésente, pas comme une technologie futuriste, mais comme un outil de productivité aussi banal qu'Excel.

Le contraste a été saisissant à mon retour en France. Malgré notre excellence technologique et notre écosystème d'innovation reconnu, j'ai découvert un gap considérable entre le \textbf{potentiel théorique} de l'IA et son \textbf{adoption effective} dans le tissu économique français. Cette observation m'a conduit à créer Luwai en 2025, avec pour mission de transformer les entreprises françaises d'\textit{"AI-curious"} à \textit{"AI-productive"}.

Cette thèse documente ce parcours entrepreneurial tout en analysant les mécanismes profonds qui expliquent les résistances à l'adoption de l'IA en France. Elle s'appuie sur \textbf{63 entretiens} menés avec des prospects et clients, \textbf{5 propositions commerciales} réelles, et l'expérience concrète de construction d'un modèle d'affaires dans ce secteur émergent.

L'angle adopté est résolument \textbf{entrepreneurial} : comment transformer les résistances identifiées en opportunités de création de valeur ? Comment construire un pont entre l'innovation technologique et les besoins opérationnels des dirigeants français ?

Au-delà de l'analyse académique, cette thèse se veut un guide pratique pour les entrepreneurs souhaitant s'engager dans l'accompagnement à la transformation par l'IA, ainsi qu'un outil de réflexion pour les dirigeants de PME-ETI confrontés à ces enjeux.
