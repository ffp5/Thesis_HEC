\chapter{Revue de Littérature et Cadre Théorique}
\label{chap:literature_review}

L'intégration de l'IA en entreprise s'inscrit dans une histoire de l'étude de l'acceptation par les utilisateurs de technologies de rupture. Cette revue de littérature explore les théories à la base de l'adoption des TIC, l'entrepreneuriat spécifique des technologies de l'IA, les particularités culturelles françaises, et les tendances du marché des services liés à la transformation digitale. Elle vise à construire un cadre scientifique solide pour analyser le paradoxe français de l'IA et situer le cas Luwai dans la recherche mondiale sur l'entrepreneuriat.

\section{Adoption Technologique et Transformation Digitale}

\subsection{Modèles Classiques d'Adoption Technologique}

Les modèles théoriques d'adoption technologique constituent le socle conceptuel pour comprendre les mécanismes d'acceptation de l'IA en entreprise. Le \textbf{Technology Acceptance Model (TAM)} de Davis \cite{davis1989perceived} reste le cadre de référence le plus utilisé dans la littérature académique \cite{artimon2025theorie}. Ce modèle postule que l'intention d'utiliser une technologie dépend de deux facteurs principaux :
\medskip
\begin{itemize}
    \item \textbf{L'utilité perçue} (Perceived Usefulness) : degré auquel une personne croit qu'utiliser une technologie améliorera ses performances professionnelles
    \item \textbf{La facilité d'usage perçue} (Perceived Ease of Use) : degré auquel une personne croit que l'utilisation d'une technologie sera sans effort
\end{itemize}
\medskip
Dans le monde de l’IA, ces variables sont à redéfinir : le concept d’utilité peut-être de haut niveau - on observe en entreprises des gains de productivité sur les tâches cognitives et routinières de 20-40\% - mais la facilité d’usage est à travailler du fait des technologies trop complexes pour le commun des mortels, et du fait de l’absence de vrai cursus de formation autour des IA.\cite{psicosmart2024resistance}.
\newpage
La \textbf{Théorie Unifiée de l'Acceptation et de l'Utilisation de la Technologie (UTAUT)} de Venkatesh et al. \cite{venkatesh2003user} enrichit substantiellement le modèle TAM en intégrant quatre déterminants clés qui s'avèrent particulièrement pertinents pour l'IA :

\begin{enumerate}
    \item \textbf{Performance Expectancy} : attente de gains de performance, cruciale pour l'IA où les bénéfices sont souvent promis mais difficiles à quantifier ex-ante
    \item \textbf{Effort Expectancy} : effort anticipé pour maîtriser la technologie, dimension critique pour l'IA générative qui nécessite l'apprentissage du "prompting" efficace
    \item \textbf{Social Influence} : influence de l'environnement social, particulièrement importante dans les PME-ETI où les décisions sont souvent prises collectivement
    \item \textbf{Facilitating Conditions} : conditions facilitantes organisationnelles, déterminantes pour l'IA qui requiert infrastructure, données et gouvernance adaptées
\end{enumerate}
\medskip
Cette approche multifactorielle s'avère particulièrement pertinente pour analyser l'adoption de l'IA en PME-ETI, où les conditions facilitantes (formation, support technique, gouvernance) jouent un rôle crucial dans la réussite des implémentations \cite{vorecol2024resistance}.
\medskip
\textbf{Extension aux modèles comportementaux récents :} Les études récentes montrent que les modèles usuels doivent être adaptés pour refléter les particularités de l’IA.  
Le \textbf{modèle AIRAM} (Artificial Intelligence Readiness and Adoption Model) conçu par Chen et al. \cite{chen2024airam} inclut quatre dimensions supplémentaires qui sont importantes pour l’IA : la maturité des données, l’agilité organisationnelle, la préparation éthique, et la capacité d’explicabilité.  
Ces dimensions font sens également dans le contexte français où le cadre normatif (\textit{IA Act}, RGPD) a une forte influence sur les décisions d’adoption.

\subsection{Spécificités de l'IA comme Technologie Disruptive}

L’intelligence artificielle présente des caractéristiques qui la différencient fondamentalement des technologies de l’information et de la communication et rendent plus complexe son adoption par des mécaniques qui ne sont pas anticipées par les modèles classiques : nous en décrivons six principaux à la lecture de la littérature récente et de nos observations terrain :
\\\\
\textbf{La "Black Box" et l'explicabilité} : L'IA générative étant peu explicable, les utilisateurs la craignent et y résistent \cite{fountaine2019building}. Elle est opaque là où les outils informatiques classiques sont, du moins partiellement, "comprenables" par leurs utilisateurs. Dans nos entretiens, 67\% des dirigeants confirment que cette opacité est selon eux la principale barrière à l'adoption.
\\\\
\textbf{L'évolutivité rapide et l'obsolescence perçue} : Les progrès fulgurants en matière d'IA (un nouveau paradigme technologique tous les six à huit mois environ) génèrent un phénomène de "technophile angoissé" chez les utilisateurs potentiels, qui redoutent de se tourner vers des solutions qui deviendront vite dépassées \cite{ransbotham2023expanding}. À cela s'ajoute l'accélération des cycles d'évolution IT propre aux PME et ETI (de trois à cinq ans).
\\\\
\textbf{L'ambiguïté des cas d'usage} : En effet, la polyvalence de l'IA est un véritable frein à son adoption. En effet, contrairement à une solution métier, dont les fonctionnalités sont clairement définies, l'IA générative a des applications dans presque tous les domaines, ce qui rend particulièrement difficile de cibler les cas d'usage prioritaires et de justifier un retour sur investissement \cite{dwivedi2021artificial}.
\\\\
\textbf{Les enjeux éthiques et réglementaires spécifiques} : La réglementation européenne sur l'intelligence artificielle et les préoccupations croissantes en matière de protection des données (RGPD) ajoutent un aspect que n'ont pas d'autres technologies : une complexité supplémentaire en termes de règles à respecter \cite{bertolucci2024artificial}. Cette dimension réglementaire l'est d'autant plus en France où la peur du gendarme occupe une place importante dans le processus décisionnel concernant les investissements technologiques.
\\\\
\textbf{L'effet de réseau et la dépendance aux données} : L'efficacité de l'IA dépend critique de la qualité et du volume des données disponibles, créant un cercle vicieux pour les organisations dont les données sont peu structurées. Cette dépendance contraste avec les logiciels traditionnels qui peuvent être efficaces indépendamment de la maturité data de l'organisation.

\subsection{Facteurs Organisationnels d'Adoption : Une Analyse Approfondie}

Les articles et ouvrages traitant l'implémentation de l'IA (cf. références du cadre théorique \cite{davis1989perceived, artimon2025theorie, psicosmart2024resistance, vorecol2024resistance, chen2024airam}) identifient plusieurs préalables organisationnels cruciaux pour une adoption effective, que notre travail de terrain modère et illustre dans le contexte des entreprises françaises :
\\\\
\textbf{Le leadership et le sponsorship exécutif} : Le soutien actif et régulier de la direction générale est parmi les facteurs clés de réussite.  Les enquêtes à l’échelle mondiale indiquent que les projets IA soutenus par un sponsor au plus haut niveau exécutif ont 3,5 fois plus de chances d’aboutir \cite{capgemini2024ai_france}.  En France, cela se confirme voire s’amplifie car notre pays, en étant caractérisé par une culture d’entreprise particulièrement hiérarchisée où il est habituel que le changement vienne du haut de la pyramide, il est nécessaire que les projets innovants obtiennent un feu vert venu “du sommet” pour démarrer réellement.
\\\\
\textbf{La culture organisationnelle et la capacité d'expérimentation} : Les sociétés innovantes et en quête de nouvelles solutions sont celles qui s’approprient le plus facilement l’IA.  
À l’inverse, les entreprises aux cultures fortes de contrôle et de conformité, que l’on retrouve souvent dans les secteurs traditionnels français, se heurtent à des résistances bien ancrées \cite{teece2007dynamic}.  Nous constatons par ailleurs que les entreprises ayant déjà connu une transformation digitale (ERP, CRM) ont tendance à adopter l’IA 2,3 fois plus que les autres.
\\\\
\textbf{Les compétences internes et la stratégie de montée en compétences} : Le manque de compétences en intelligence artificielle (IA) en interne est l'un des principaux obstacles à l'utilisation de l'IA, notamment dans les PME/ETI, où les budgets de formation sont limités \cite{bpifrance2025ia}.  Cette difficulté est amplifiée en France par la rareté des talents de l'IA (le déficit de profils est estimé à 15 000 par France Stratégie \cite{france_strategie2025make}) et la concurrence des grands groupes et des startups de la tech.
\\\\
\textbf{La gouvernance des données et la maturité informationnelle} : La réussite des projets d’intelligence artificielle repose sur une gouvernance des données solide, ce qui constitue souvent un point de rupture \cite{wang2024data_governance}. Nos entretiens montrent que 73\% des PME-ETI interrogées n’ont pas de politique data formalisée ce qui bride leur capacité à exploiter l’IA.
\\\\
\textbf{L'écosystème de partenaires et la capacité d'orchestration} : Pour qu’une entreprise tire tout le parti du potentiel de l’intelligence artificielle, il faut souvent faire appel à un réseau de partenaires (technologiques, conseil, intégration). Or, les PME-ETI françaises, qui ont l’habitude de travailler en mode fournisseur-client classique, ont du mal à développer des compétences qui leur permettent d’animer un réseau de plusieurs partenaires.

\section{Innovation et Entrepreneurship Technologique}

\subsection{Innovation Disruptive et IA : Une Relecture Contemporaine}

Pour réussir à déployer l’IA, il est souvent nécessaire de s’appuyer sur un réseau de partenaires (technologiques, conseil, intégrateurs). Les PME-ETI françaises, habituées à des relations fournisseur-client bilatérales, ont souvent du mal à développer ces compétences de gestion multi-partenaires.
\\\\
La théorie des ruptures de Christensen constitue un cadre d’analyse utile pour comprendre l’impact de l’IA sur les entreprises des secteurs traditionnels, même si cette théorie mérite d’être retravaillée pour prendre en compte les spécificités des IA génératives.
\\\\
Il est vrai que l’IA a plusieurs traits de l’innovation disruptive :
\begin{itemize}
    \item \textbf{Performance initialement inférieure} dans certains domaines critiques : qualité variable des outputs, fiabilité encore limitée, biais algorithmiques
    \item \textbf{Amélioration rapide} des performances techniques selon une courbe exponentielle (loi de Moore appliquée à l'IA)
    \item \textbf{Nouvelle proposition de valeur} basée sur l'accessibilité (démocratisation via les interfaces conversationnelles) et le coût (marginal tendant vers zéro)
    \item \textbf{Menace progressive pour les acteurs établis} dans les services intellectuels traditionnels (conseil, audit, rédaction, analyse)
\end{itemize}
\medskip
Ce schéma explique la résistance que nous avons pu constater auprès des entreprises de services à l’instar des cabinets de conseil, d’audit ou d’expertise-comptable dont il est fortement questionné les modèles économiques reposant sur la vente de temps-conseil par la mise en œuvre de solutions d’automatisation des processus cognitifs \cite{syntec2024ai}.
\\\\
\textbf{Le dilemme de l'innovateur appliqué à l'IA} : Les entreprises traditionnelles se retrouvent souvent confrontées à la nécessité de jongler entre l’exploitation de leurs compétences actuelles et l’exploration de nouvelles voies, notamment celles offertes par l’intelligence artificielle (IA). Cette dualité est particulièrement frappante dans les entreprises françaises de services intellectuels : en effet, l’IA est à la fois un levier pour accroître la productivité des consultants habituels et un risque pour la pérennité d’un modèle économique reposant sur une facturation horaire.
\\\\
\textbf{Spécificités de l'IA vs innovations disruptives traditionnelles} : Par rapport aux innovations disruptives traditionnelles qui voient leur qualité s'améliorer de façon linéaire, l'IA est une innovation qui connaît des améliorations par paliers (breakthrough moments) rendant particulièrement difficile la prédiction de son évolution. Cette incertitude a un impact sur la manière dont les entreprises françaises, culturellement réticentes à la prise de risque, envisagent adopter l'IA.

\subsection{Entrepreneurship et Accompagnement Technologique : Nouveaux Modèles}

L’avènement de l’intelligence artificielle ouvre de nouvelles perspectives à l’entrepreneuriat, notamment dans le domaine du conseil et de l’accompagnement où s’inscrit Luwai. De ce que nous avons observé et de ce que nous avons compris de l’écosystème français, plusieurs business modèles sont en train de voir le jour :
\\\\
\textbf{Les "IA Enablers" ou facilitateurs d'adoption} : Des startups qui développent des interfaces simplifiées et des services d'accompagnement pour rendre l'IA accessible au plus grand nombre. Ces startups se situent entre les pure players technologiques (OpenAI, Microsoft) et les utilisateurs finaux et créent de la valeur en traduisant et adaptant\cite{parker2016platform}.
\\\\
\textbf{Les intégrateurs sectoriels verticaux} : Entrepreneurs qui développent des solutions d'IA pour des secteurs spécifiques (legal tech, med tech, fintech) avec une approche verticale leur permettant d'avoir une expertise du métier et des cas d'usage très spécifiques.
\\\\
\textbf{Les services d'accompagnement hybrides} : Des consultants et des formateurs experts en conduite du changement en IA, tels que Luwai avec son modèle de formation, conseil et delivery.
Cela correspond très bien au marché français : préférence pour l'accompagnement par un intervenant vs. les solutions de formation en ligne ou "self-service".
La littérature entrepreneuriale récente évoque la \textit{customer discovery} comme un processus clé des startups de la deeptech, au sein desquelles les besoins clients évoluent vite et les solutions à trouver sont nombreuses \cite{blank2013lean, osterwalder2014value}.  
Luwai s’inscrit pleinement dans ce mouvement puisqu’il a opéré 3 pivots importants sur son business model en moins de 3 mois.

\subsection{Dynamic Capabilities et Transformation IA : Framework Appliqué}

L’approche des \textbf{Dynamic Capabilities} de Teece s’avère très utile pour comprendre la transformation de l’intelligence artificielle dans les entreprises françaises \cite{teece2007dynamic}.  Ces capacités dynamiques se traduisent par trois processus clés que nous illustrons ici avec l’écosystème IA :

\begin{enumerate}
    \item \textbf{Sensing} (Détection) : Capacité à identifier les opportunités et menaces IA dans l'environnement concurrentiel et technologique. Cette dimension inclut la veille technologique, l'évaluation des cas d'usage pertinents, et l'analyse des mouvements concurrentiels.
    
    \item \textbf{Seizing} (Saisie) : Capacité à saisir ces opportunités via l'investissement stratégique et le développement de compétences. Cela englobe les décisions d'allocation budgétaire, la sélection de partenaires, et la définition de priorités d'implémentation.
    
    \item \textbf{Reconfiguring} (Reconfiguration) : Capacité à reconfigurer les actifs, processus et structures organisationnelles pour intégrer l'IA de manière optimale. Cette dimension est souvent la plus complexe car elle implique des changements organisationnels profonds.
\end{enumerate}
\medskip
Force est de constater que les PME-ETI françaises présentent des faiblesses marquées dans ces trois dimensions, ce qui explique leur propension à une adoption laborieuse, voire chaotique, de l'IA :
\\\\
\textbf{Déficit de Sensing} : 68 \% des entreprises interrogées ont déclaré que leurs sociétés n’étaient pas dotées de processus de veille artificielle « bien formalisés » et qu’elles s’appuyaient en grande partie sur des sources de veille non structurées (presse généraliste, réseaux sociaux professionnels). Cela leur donne une vision parcellaire des opportunités.
\\\\
\textbf{Faiblesse en Seizing} : Nous avons constaté, à travers nos entretiens, que les cycles de décision des PME-ETI françaises sont longs et très hiérarchisés (3 à 5 niveaux de validation). Or, ils ne sont pas adaptés à la vitesse de changement que présente l'IA pour saisir les opportunités. Dans la plupart des cas, il faut entre 6 et 9 mois, selon nos interlocuteurs, pour prendre des décisions. Nos cycles de décision observés en entretiens sont donc inadaptés à la vitesse de la technologie.
\\\\
\textbf{Résistance au Reconfiguring} : La manière traditionnelle de travailler des entreprises françaises, caractérisée par une forte préférence pour le statu quo, se reflète également dans des cas comme celui-ci. Souvent, les entreprises optent pour des solutions d'IA qui peuvent s'intégrer dans leur manière de fonctionner plutôt que de chercher à revoir profondément leur manière de travailler.

\section{Spécificités Culturelles et Organisationnelles Françaises}

\subsection{Culture Nationale et Adoption Technologique : Analyse Hofstedienne Appliquée}

Les études de Hofstede sur les dimensions culturelles nationales \cite{hofstede2001culture} fournissent un cadre utile pour comprendre comment les valeurs profondément ancrées dans la culture française peuvent influencer la manière dont l'IA est adoptée au niveau local. Trois dimensions sont à cet égard particulièrement significatives et peuvent aider à mieux comprendre les freins identifiés :
\\\\
\textbf{Distance au pouvoir élevée (68 vs 40 moyenne mondiale)} : La France est une société politique et économique très structuree en termes de hiérarchie de commandement ou influence, où il est parfois difficile d'adopter des solutions prises en bas de l'échelle comme des outils technologiques tels que l'IA générative qui, par leur nature décentralisée et démocratisante, pourraient sembler subversives aux hiérarchies traditionnelles basées sur l'information \cite{meyer2014culture}. C'est sans doute pour cela que l'on voit que l'IA a parfois un chemin plus difficile pour s'imposer dans les entreprises françaises que dans celles des pays anglophones.
\\\\
\textbf{Aversion à l'incertitude forte (86 vs 65 moyenne mondiale)} : Cela explique la forte propension qu'ont nos compatriotes à préférer les dispositifs réglementaires (comme l'IA Act européen, les projets étatiques...) et à se montrer circonspects à l'égard des technologies émergentes dont elles ne perçoivent pas encore parfaitement les enjeux de long terme \cite{bertolucci2024artificial}. Cette méfiance s'exprime en effet par des exigences de garanties, de mise en responsabilité, de retour en arrière possible sur les solutions IA qui peuvent rallonger les délais pour trancher une adoption.
\\\\
\textbf{Individualisme modéré (71 vs 91 États-Unis)} : Relativement plus faible qu'aux Etats-Unis, cette dimension favorise les approches de formation "en groupe" et d'adoption de la technologie "en groupe", ce qui fait que des modèles d'accompagnement "en groupe" vs "individuel" ont plus de chances de réussir. C'est à mon sens la raison pour laquelle il est plus difficile de trouver des succès stories quand on regarde des start-up de la Edtech comme Luwai qui font beaucoup de formation "individuelle" vs "en groupe". 
\\\\
\textbf{Orientation long terme et pragmatisme français} : La particularité française en termes de management, où le long terme l'emporte généralement sur la recherche de solutions rapides et/ou de compensation de court terme, explique que les critères de sélection des projets IA sont différents. Les business leaders français privilégient souvent les investissements IA qui s'insèment dans une vision de transformation profonde et pas seulement dans une logique d'optimisation de court terme.

\subsection{Modèle Français vs Modèle Anglo-Saxon : Analyse Comparative}

En comparant l'écosystème américain à celui de la France, il est possible d'identifier des particularités structurelles qui conditionnent les approches en matière d'adoption et ouvrent certaines portes aux acteurs locaux:
\\\\
\textbf{Rapport au risque et culture de l'échec} : La mentalité française "droit à l'erreur mais... à la marge" semble - tout du moins à première vue - s'opposer au phénomène de start-up américain dit "fail fast, learn fast"\cite{meyer2014culture}. Cela se traduit dans les faits par une préférence pour des pilotes "proof of concept" longs et poussés, plutôt que des démarrages rapides et fondés sur une dynamique d'échange et d'ajustement continu. A titre d'exemple, des durées moyennes de phase pilote de 4 à 6 mois en France contre 2 à 4 semaines dans des start-ups américaines.
\\\\
\textbf{Rôle de l'État et interventionnisme public} : La démarche proactive de l’État français en faveur de l’Intelligence Artificielle (œuvres stratégiques nationales, intervention de BPIFrance, régulation qui tend à favoriser le développement de l’IA…) tranche avec l’approche plus libérale des américains selon lesquels il y a lieu de laisser jouer la concurrence sur le marché \cite{france_strategie2025make}. Cette opposition entre deux mod è les a des conséquences profondes quant aux modalités d’adoption de l’IA, lesquelles sont plus encadrées et appuyées en France où il peut y avoir des fa çons de faire affaires différentes de l’habitude de tout entrepreneur US pour le meilleur et pour le pire
\\\\
\textbf{Écosystème entrepreneurial et philosophie d'accompagnement} : En France, on a une préférence pour l'accompagnement et la formation (avec l'intervention de BPI France, le rôle des incubateurs, des CCI, etc.) alors que les Américains sont plus dans l'action, la réalisation de "deals". Cela veut dire que le marché français est plus propice à des modèles hybrides où l'accompagnement par l'humain est la clé que des pure players de la tech.
\\\\
\textbf{Temporalité des décisions et processus collectifs} : Les processus de décision français, qui sont habituellement plus longs et visent le consensus, semblent paradoxalement mieux convenir aux projets d’IA. En effet, ces projets requièrent une réflexion stratégique et l’adhésion d’un collectif. Ce qui est souvent vu comme un inconvénient peut être, en fait, un atout pour l’adoption dans la durée et pour une appropriation en profondeur de la solution d’IA.

\subsection{PME-ETI Françaises : Caractéristiques Structurelles et Enjeux Spécifiques}

Les petites et moyennes entreprises ont une place prépondérante dans l'économie française : en effet, elles représentent 99,8\% des entreprises et environ 63\% des emplois privés. Or, ces entreprises ont une approche de l’intelligence artificielle qui dépendra de leur visibilité à l’international, d’une pondération entre relations clients et fournisseurs, de leur degré de transformation numérisée, du secteur dans lequel elles évoluent, de leurs effectifs et de leurs ressources humbles ou financières\cite{bpifrance2025ia, france_strategie2025make} :
\\\\
\textbf{Contraintes de ressources et arbitrages d'investissement} : Les petites et moyennes entreprises ont peu de budgets et du temps pour tester des nouvelles technologies, c’est pourquoi il est important qu’elles puissent avoir des solutions clés en main, et qu’elles soient bien accompagnées. Psychologique, cela se comprend par la préférence structurelle (enfin chez les "mid-market") pour les solutions "service" plutôt que d’acquérir une compétence en interne pour aboutir à un segment de marché essentiellement captif pour des prestataires comme Luwai.
\\\\
\textbf{Influence prépondérante du dirigeant-propriétaire} : Le dirigeant-propriétaire est une personnalité-clé. C’est sa vision de l’innovation, son aisance avec les outils numériques et sa perception des apports de l’intelligence artificielle qui concoivent le plus souvent l’adoption de telles technologies au sein des PME-ETI françaises \cite{bpifrance2017dirigeants}. En cela, nous comprenons bien que le fait que la décision "revienne" à un seul homme ou à une seule femme peut être une chance pour l’entreprise (la conviction du dirigeant et le feu vert à la décision attendu) comme un obstacle (la décision dépendant de la seule et unique résistance du dirigeant).
\\\\
\textbf{Culture de proximité et relations humaines privilégiées} : Les PME-ETI françaises sont de grands clients fidèles qui privilégient des relations de confiance sur le long terme avec des fournisseurs/entrepreneurs locaux. En conséquence, les relations business-tobusiness y sont souvent qualitatives, voire même relationnelles, dans le sens ou les clients sont accompagnés au-delà de leur simple consommation de produits et services. Il y a peu de place pour les « solutions Silicon Valley », et pour les prestataires qui ne seraient pas capables de s'inscrire dans une dynamique gagnant-gagnant durable. \cite{sage2025pme_transformation} En cela, les entrepreneurs français sont servis par leur image de marque (soft power), à condition bien sûr qu'ils soient en mesure d'offrir de la proximité, du sur-mesure aux petits soins pour leurs clients, et aussi des prix compétitifs.
\\\\
\textbf{Pression concurrentielle modérée et adoption attentiste} : De manière générale, les PME-ETI évoluent dans des secteurs où la compétition est moins féroce que dans des environnements hyper-concurrentiels, tels que la tech ou la finance. Dans ces secteurs matures, la pression concurrentielle est telle qu'elles peuvent se permettre d'adopter une posture attentiste vis-à-vis de l'innovation. Ainsi, ces périodes plus longues peuvent, paradoxalement, conduire à une adoption de l'intelligence artificielle plus réfléchie et plus durable.

\section{Services Professionnels et Conseil en Transformation}

\subsection{Évolution du Marché du Conseil en France : Mutations Structurelles}

Le marché français du conseil (15,2 milliards d'euros en 2024 d'après Syntec Conseil) est bouleversé par la digitalisation et la montée de l'IA \cite{syntec2024ai}. Plusieurs évolutions de fond redessinent le jeu concurrentiel :
\\\\
\textbf{Fragmentation croissante de la demande} : Les demandes d’accompagnement sur les sujets IA sont plus précises, plus sophistiquées et plus techniques que celles adressées par des cabinets de conseil traditionnel, ce qui nourrit une forte dynamique de boutiques spécialisées face aux grands cabinets généralistes. Cette tendance se confirme avec la progression des opérations confiées aux acteurs spécialisés : selon Syntec, 45\% des missions IA sont aujourd’hui confiées à des boutiques spécialisées contre 28\% en 2022.
\\\\
\textbf{Hybridation formation-conseil-technologie} : La nature technique complexe de l’intelligence artificielle conduit à une forte demande de développement des compétences, en parallèle des missions de conseil classique, donnant lieu à de nouveaux modèles hybrides qui mêlent formation, conseil stratégique et livraison de solution. Cette tendance interroge la séparation historique entre les acteurs de la formation et les consultants.
\\\\
\textbf{Pression sur les modèles économiques traditionnels} : L’émergence de l’IA grand public exerce une pression à la baisse sur les tarifs des prestations intellectuelles classiques (recherche, synthèse, rédaction), obligeant à se repositionner vers des prestations à plus forte valeur ajoutée (stratégie, créativité, relation client) \cite{mckinsey2023consulting_ai}.
\\\\
\textbf{Verticalisation sectorielle accélérée} : Les spécialisations par secteur deviennent incontournables pour être réellement utiles dans un contexte où foisonnent les outils d'IA généralistes. Ainsi, les cabinets développent des compétences verticales (industrie 4.0, santé digitale, fintech) pour se démarquer des solutions uniformisées.
\\\\
\textbf{Internationalisation des standards et des pratiques} : L'IA étant un phénomène global, les standards et meilleures pratiques s'internationalisent rapidement, réduisant l'avantage concurrentiel des spécificités locales traditionnelles du conseil français.

\subsection{L'IA dans le Secteur Public Français : Laboratoire d'Expérimentation}

L’intégration de l’IA dans l’administration française connaît un parcours jalonné d’apprentissage s’adressant à la fois au secteur public et à un écosystème d’acteurs privés. Cette composition se singularise par six patterns que met en lumière Bertolucci \cite{bertolucci2024artificial}:
\\\\
\textbf{Résistances institutionnelles et bureaucratiques} : Les administrations françaises, qui relèvent d’une forte hiérarchie et où les choses se passent selon des procédures très codifiées, sont confrontées aux mêmes résistances que les PME-ETI classiques. Ces résistances ne sont pas propres au secteur public mais trahissent des caractéristiques culturelles des organisations françaises plus largement.
\\\\
\textbf{Exigences de transparence et d'explicabilité} : La demande d'explicabilité des systèmes d'intelligence artificielle (IA) utilisés dans le secteur public met en évidence les limites de l'acceptation des systèmes d'IA "boîte noire" par la société.
Cette problématique est importante dans les secteurs privés réglementés (banque, assurance, santé, etc.) où elle oriente les choix technologiques et les stratégies de déploiement.
\\\\
\textbf{Gouvernance des données publiques comme modèle} : La manière dont est gérée la data dans le secteur public (conformité au RGPD, sécurité de la data, usages et éthique) revient à constituer des référentiels de travail et des sources d'inspiration pour des entreprises privées, particulièrement pour les PME et les ETI qui partent de loin en matière de gouvernance des données.
\\\\
\textbf{Approche "compliance-first" vs "innovation-first"} : En France, la manière de procéder du secteur public est plutôt "conformité d'abord" qui sécurise la mise en place mais peut freiner l'innovation. Cela a pour conséquence de s'infiltrer dans le privé où l'on retrouve parfois les mêmes comportements, contrairement à l'approche anglo-saxonne "innovation-first".

\subsection{Business Models Émergents dans l'Accompagnement IA}
L’accompagnement en IA est source de nouveaux business-models hybrides qui bousculent les codes. À partir de notre observation du paysage français, nous en avons identifié cinq :
\\\\
\textbf{Formation + Conseil + Delivery (Modèle Full-Service)} : Offre allant de la sensibilisation à la stratégie puis à la mise en œuvre, du début à la fin. C’est le modèle vers lequel Luwai a rebasculé après plusieurs allers-retours avec le marché. Celui-ci correspond bien à l’attente hexagonale d’un seul interlocuteur et d’une prise en main totale.
\\\\
\textbf{SaaS + Services (Modèle Tech + Services)} : Association entre une plateforme technologique propriétaire et une prestation d’accompagnement menée par des humains, plébiscitée par la majorité des startups IA B2B en France. Cela permet d’avoir en même temps une offre scalable exploitant la force de la technologie et une personnalisation par le service humain.
\\\\
\textbf{Community + Consulting (Modèle Communautaire)} : Construction d’une communauté d’utilisateurs \& d’experts à même de générer des leads entrants pour des offres premium de conseil. Ce modèle utilise l’effet réseau et la dynamique communautaire qui sont chers aux Français.
\\\\
\textbf{Subscription + Support (Modèle Récurrent)} : Modèle récurrent sur la base d’une offre d’accès à des ressources (templates, playbooks, formations, etc.) et sur le soutien apporté ponctuellement. Cela permet de lisser ses revenus et de pérenniser la relation client.
\\\\
\textbf{Marketplace + Curation (Modèle Plateforme)} : Marketplace mettant en relation des entreprises et des consultants IA spécialisés, avec une dimension curation et garantie qualité. Ce modèle répond à une expertise parfois très fragmentée et à un besoin de réassurance des PME-ETI.
\section{Recherche Récente sur l'Adoption de l'IA en Entreprise}

\subsection{Études Empiriques Internationales et Spécificités Contextuelles}

Les études actuelles sur l’adoption de l’IA en entreprise montrent des usages complexes qui diffèrent sensiblement selon les dimensions géographiques, sectorielles ou organisationnelles. Wagner et al. \cite{wagner2022artificial} analysent que l’IA a transformé jusqu’aux méthodes mêmes pour conduire des investigations, en soulignant combien l’IA peut transformer les fondements des paradigmes de recherche.
\\\\
L’enquête longitudinale de MIT Sloan Management Review \cite{ransbotham2023expanding}, portant sur plus de 3000 cas, révèle un “paradoxe intention-action” des entreprises très marqué : si 73\% des sociétés européennes jugent l’IA importante voire critique pour leur stratégie, seules 28\% ont porté des projets jusqu’à des déploiements à l’échelle de l’organisation. Ce “gap d’implémentation” est d’ailleurs plus important en France où les sociétés ambitionnent de multiplier les démarches pilotes et interventions expérimentales sans déboucher forcément sur des déploiements à l’échelle industrielle.
\\\\
\textbf{Facteurs explicatifs du gap français} : L'analyse comparative internationale révèle plusieurs facteurs spécifiques au contexte français :
\begin{itemize}
    \item \textbf{Perfectionnisme culturel} : Tendance à prolonger les phases pilotes jusqu'à obtention de résultats "parfaits" avant scaling
    \item \textbf{Aversion au risque institutionnelle} : Préférence pour les validations multiples et les consensus larges avant déploiement
    \item \textbf{Complexité organisationnelle} : Structures décisionnelles multi-niveaux rallongeant les cycles d'adoption
\end{itemize}

\subsection{Spécificités de l'Écosystème Français : Analyse Différentielle}

La plupart des modèles classiques d’adoption des technologies (TAM, UTAUT) ont du mal à expliquer les spécificités comportementales et organisationnelles liées à l’IA. Plusieurs auteurs ont proposé des extensions théoriques particulièrement adaptées :

\textbf{Le modèle AIRAM (Artificial Intelligence Readiness and Adoption Model)} de Chen et al. conserve les quatre facteurs traditionnels, mais les complète par quatre dimensions liées à l’IA :

\begin{itemize}
    \item \textbf{Technical Readiness} : L’équipement de base pour du software, APIs, des capacités de calcul, de stockage et d’intégration.
    \item \textbf{Data Maturity} : L’organisation des données ainsi que la gouvernance et l’accessibilité de ces données pour tous au sein de l’entreprise.
    \item \textbf{Organizational Agility} : L’adaptation rapide, la culture du test and learn, l’agilité des processus.
    \item \textbf{Ethical Preparedness} : L’établissement de règles éthiques, la conformité réglementaire, la gestion des biais et l’explicabilité.
\end{itemize}
\medskip
Ce modèle paraît particulièrement pertinent pour décortiquer les résistances observées au sein des PME-ETI françaises, où la dimension «Data Maturity» est souvent le principal facteur limitant, suivie de «Organizational Agility».
\\\\
\textbf{Le modèle AIDA (AI Diffusion and Adoption)} proposé par Kumar et al. porte spécifiquement sur la façon dont l’IA se propage dans des écosystèmes organisationnels. Ce modèle identifie quatre phases de l’adoption :

\begin{enumerate}
    \item \textbf{Awareness Phase} : Découvrir l’IA et ses opportunités.
    \item \textbf{Interest Phase} : Explorer des cas d’usage pour évaluer l’intérêt.
    \item \textbf{Decision Phase} : Prendre des décisions sur les investissements à réaliser, les solutions à adopter.
    \item \textbf{Action Phase} : Mettre en œuvre, déployer et diffuser l’IA dans l’entreprise.
\end{enumerate}

Sur le terrain, nous constatons que les PME-ETI françaises semblent souvent coincées entre les phases Interest et Decision, ce qui implique qu’il peut être nécessaire de mettre en place des actions spécifiques pour débloquer la situation et faire avancer les choses.

\section{Synthèse du Cadre Théorique et Positionnement de la Recherche}

\subsection{Gaps Identifiés et Contributions Attendues}

Cette revue de littérature exhaustive révèle plusieurs gaps théoriques et empiriques significatifs que cette recherche ambitionne de combler, positionnant notre contribution dans le paysage académique international :
\\\\
\textbf{Gap empirique majeur} : Il n’existe pas d’études qualitatives assez profondes sur l’adoption de l’IA dans les PME-ETI françaises, segment pourtant vital pour l’économie nationale (63\% de l’emploi privé). La plupart des recherches se concentrent sur les grandes entreprises ou les jeunes entreprises innovantes, ne permettant pas de se pencher sur cet entre-deux si important.
\\\\
\textbf{Gap théorique fondamental} : On ne peut pas coller sans plus à l’utilisation de l’IA les modèles «classiques» d’adoption de décisions technologiques (TAM, UTAUT), ceux-ci ont besoin d’être bien retouchés pour coller à ce cas particulier et à la culture française. Les nouvelles tentatives (AIRAM, AIDA) sont testées surtout dans des pays anglo-saxons; ce mouvement les hasarde un peu pour la France.
\\\\
\textbf{Gap méthodologique critique} : On ne trouve pas assez de manières de chercher – ancrées dans la rigueur académique mais aussi dans la «réalité» de l’entreprise – pour voir les phénomènes de ralliement voire d’adhésion à l’IA en même temps.
\\\\
\textbf{Gap pratique et managérial} : On manque des méthodes consacrées, des guides précis pour les entrepreneurs voulant montrer un chemin par lequel il leur serait possible, dans leur secteur et à la faveur de l’IA, de trouver un business model qui vaille la peine.

\subsection{Positionnement Théorique de la Recherche Luwai}

Cette recherche se positionne à l'intersection de plusieurs champs disciplinaires, créant une contribution théorique originale :
\\\\
\textbf{Contribution aux théories d'adoption technologique} : Extension des modèles TAM/UTAUT au contexte spécifique de l'IA générative et aux spécificités culturelles françaises, avec développement d'un modèle d'adoption "à la française" privilégiant l'accompagnement collectif vs l'expérimentation individuelle.
\\\\
\textbf{Contribution à l'entrepreneurship technologique} : Développement du framework "Formation-Conseil-Delivery" comme archétype de modèle d'affaires adapté aux services B2B technologiques émergents, avec analyse des mécanismes de scaling et de personnalisation simultanés.
\\\\
\textbf{Contribution aux études culturelles et managériales} : Analyse des spécificités françaises d'adoption technologique au prisme de l'IA, complétant les travaux hofstédiens par des observations contemporaines et sectorielles.
\\\\
\textbf{Contribution méthodologique} : Validation de l'approche "observation participante entrepreneuriale" comme méthode de recherche hybride combinant immersion terrain et analyse académique, particulièrement adaptée aux phénomènes émergents et évolutifs.

\subsection{Implications pour les Parties Suivantes}

Ce cadre théorique établit les fondations conceptuelles pour l'analyse empirique qui suit :

\textbf{Pour le diagnostic terrain (Partie IV)} : Les modèles théoriques présentés fournissent les grilles d'analyse pour interpréter les 63 entretiens menés et identifier les patterns de résistance et d'adoption spécifiques au contexte français.

\textbf{Pour le cas d'étude Luwai (Partie V)} : Le framework entrepreneurial et les modèles d'adoption permettent d'analyser l'évolution du modèle d'affaires Luwai et d'en extraire des enseignements généralisables sur la construction d'entreprises dans ce secteur.

\textbf{Pour les recommandations (Partie VI)} : Les gaps identifiés et les spécificités culturelles analysées nourrissent directement les recommandations pratiques destinées aux entrepreneurs, dirigeants et décideurs publics.

Le cas Luwai, analysé dans les parties suivantes, permet d'explorer empiriquement ces gaps théoriques à travers l'expérience concrète d'un entrepreneur confronté aux réalités du terrain français. Cette approche d'observation participante offre un accès privilégié aux dynamiques d'adoption souvent invisibles dans les études traditionnelles et permet une contribution originale aux champs théoriques mobilisés.