\chapter{Introduction}
\label{chap:introduction}

\section{Contexte et enjeux}

La France se trouve aujourd'hui dans une position paradoxale face à l'intelligence artificielle. D'un côté, notre pays dispose d'atouts indéniables : un écosystème de recherche reconnu mondialement avec l'INRIA et des laboratoires d'excellence, un tissu de startups dynamique qui a produit des pépites comme Mistral AI ou Hugging Face, et une position de leadership européen dans l'encadrement éthique de l'IA avec l'IA Act.
\\\\
D'un autre côté, l'adoption effective de l'IA dans le tissu économique français reste contrastée. Selon le baromètre du numérique 2024 \cite{laboratoire2025how}, si 33\% des Français ont utilisé des outils d'IA générative, cette adoption demeure principalement personnelle et sporadique. Dans le monde professionnel, l'écart se creuse entre les grandes entreprises et les PME-ETI qui constituent pourtant l'épine dorsale de notre économie.
\\\\
Cette situation interpelle d'autant plus que les enjeux sont considérables. L'IA représente un potentiel d'\textbf{augmentation de la productivité} de 20 à 40\% sur de nombreuses tâches, selon nos observations terrain. Pour une économie française en quête de compétitivité, ces gains de performance ne peuvent être ignorés.
\\\\
Le \textbf{paradoxe français} de l'IA se manifeste à plusieurs niveaux :
\\
\begin{itemize}
    \item \textbf{Au niveau technologique :} Nous disposons d'un écosystème d'innovation de premier plan mais peinons à diffuser ces innovations dans le tissu économique.
    \item \textbf{Au niveau organisationnel :} Les entreprises françaises excellent dans l'innovation produit mais montrent des résistances culturelles à l'adoption de nouvelles méthodes de travail.
    \item \textbf{Au niveau entrepreneurial :} L'écosystème startup français est dynamique mais les services d'accompagnement peinent à adresser efficacement le segment des PME-ETI.
\end{itemize}
C'est dans ce contexte que s'inscrit la création de Luwai et l'expérience entrepreneuriale qui nourrit cette thèse. En tant qu'ingénieur de grandes écoles françaises ayant vécu l'adoption naturelle de l'IA dans la Silicon Valley puis confronté aux résistances françaises, j'ai identifié une opportunité de création de valeur dans l'accompagnement des entreprises françaises vers une utilisation productive de l'IA.

\section{Problématique centrale}

Cette thèse s'articule autour d'une question fondamentale :

\begin{quote}
\textbf{Comment expliquer l'écart entre le potentiel de l'IA et son adoption effective dans les PME-ETI françaises, et quelles stratégies entrepreneuriales permettent de transformer ces résistances en opportunités de création de valeur ?}
\end{quote}

Cette problématique centrale se décline en trois sous-questions opérationnelles :

\begin{enumerate}
    \item \textbf{Quelles sont les résistances spécifiques} à l'adoption de l'IA dans les PME-ETI françaises et comment se manifestent-elles selon les secteurs et les profils d'entreprises ?
    \item \textbf{Comment construire un modèle d'affaires viable} pour accompagner ces entreprises dans leur transformation, en naviguant entre les contraintes de scalabilité et les besoins de personnalisation ?
    \item \textbf{Quels leviers entrepreneuriaux et managériaux} permettent d'accélérer l'adoption de l'IA et de maximiser son impact opérationnel ?
\end{enumerate}

L'angle entrepreneurial adopté dans cette thèse permet d'aborder ces questions sous un prisme résolument pratique, alimenté par l'expérience concrète de construction et de développement de Luwai.

\section{Objectifs de recherche}

Cette recherche vise quatre objectifs principaux :

\subsection{Cartographier l'écosystème et la chaîne de valeur IA en France}
Identifier les acteurs clés, analyser leurs interactions et comprendre les flux de valeur dans l'accompagnement à l'adoption de l'IA, en particulier pour les PME-ETI.

\subsection{Identifier les résistances organisationnelles et culturelles spécifiques}
Développer une taxonomie opérationnelle des freins à l'adoption de l'IA, en distinguant les résistances techniques, organisationnelles, culturelles et économiques propres au contexte français.

\subsection{Analyser le modèle entrepreneurial Luwai comme cas d'étude}
Documenter et analyser l'évolution du modèle d'affaires Luwai, de sa genèse à ses pivots stratégiques, pour en extraire des apprentissages généralisables sur l'entrepreneurship dans ce secteur.

\subsection{Formuler des recommandations pour entrepreneurs et décideurs}
Proposer des frameworks pratiques et des recommandations actionnables pour les entrepreneurs souhaitant se positionner sur ce marché et les dirigeants de PME-ETI engagés dans leur transformation IA.

\section{Méthodologie}

Cette recherche adopte une \textbf{approche mixte} combinant rigueur académique et pragmatisme entrepreneurial. Elle s'appuie sur trois piliers méthodologiques complémentaires :

\subsection{Revue de littérature académique et professionnelle}
\begin{itemize}
    \item Modèles classiques d'adoption technologique (TAM, UTAUT)
    \item Littérature sur l'innovation disruptive et l'entrepreneurship technologique
    \item Analyses sectorielles et rapports professionnels sur l'IA en France
\end{itemize}

\subsection{Étude de cas entrepreneurial}
\begin{itemize}
    \item Documentation de la genèse et du développement de Luwai
    \item Analyse de l'évolution du modèle d'affaires et des pivots stratégiques
    \item Observation participante en tant que CEO-fondateur
\end{itemize}

\subsection{Collecte de données primaires}
La richesse empirique de cette recherche repose sur une importantes collecte de données menée entre juin et août 2025 :
\medskip
\begin{itemize}
    \item \textbf{500 appels prospects}  
    Menés via \textit{cold calling}, ayant abouti à 63 rendez-vous effectués (taux de conversion : 12,6\%).  
    Les secteurs représentés sont diversifiés : 
    \begin{itemize}
        \item Conseil : 32\%
        \item Industrie : 25\%
        \item Services : 21\%
        \item Tech : 15\%
        \item Finance : 7\%
    \end{itemize}

    \item \textbf{7 propositions commerciales réelles acceptées} dont 5 analysées en détail :  
    Aesio \cite{luwai2025aesio}, Antilogy \cite{luwai2025antilogy}, Intégrhale \cite{luwai2025integrhale}, Carecall \cite{luwai2025carecall}, Tectona \cite{luwai2025tectona}.
    
    \item \textbf{Observation directe}  
    Analyse des interactions commerciales, des cycles de vente et de l’évolution des besoins clients sur 9 mois d’activité.
\end{itemize}

\medskip
La méthode d’analyse combine :
\begin{itemize}
    \item le codage thématique des entretiens,
    \item l’identification des patterns récurrents,
    \item le \textit{mapping} des cas d’usage émergents.
\end{itemize}
\medskip
Cette approche permet de relier observations terrain et cadres théoriques pour produire des insights actionnables.

\section{Plan et contributions attendues}

Cette thèse s'organise en cinq chapitres principaux après cette introduction :
\bigskip
\begin{itemize}
    \item \textbf{Chapitre 2 :} Revue de littérature et cadre théorique - Fondements académiques de l'adoption technologique et spécificités du contexte français.
    \item \textbf{Chapitre 3 :} Diagnostic terrain : résistances et opportunités - Analyse empirique des freins et leviers identifiés via les entretiens.
    \item \textbf{Chapitre 4 :} Cas d'étude Luwai - Documentation du modèle entrepreneurial et de son évolution.
    \item \textbf{Chapitre 5 :} Recommandations et perspectives - Frameworks pratiques et implications pour l'écosystème.
    \item \textbf{Chapitre 6 :} Conclusion - Synthèse des apports, limites et réflexions finales.
\end{itemize}
\newpage
Les \textbf{contributions attendues} se situent à trois niveaux :
\bigskip
\begin{itemize}
    \item \textbf{Contribution empirique :} Première étude qualitative approfondie sur les résistances à l'IA dans les PME-ETI françaises, avec une taxonomie opérationnelle des freins et leviers d'adoption.
    \item \textbf{Contribution théorique :} Extension des modèles classiques d'adoption technologique au contexte spécifique de l'IA et développement d'un framework "Formation-Conseil-Delivery" pour les services B2B.
    \item \textbf{Contribution managériale :} Guide pratique d'évaluation des opportunités IA pour les dirigeants et recommandations stratégiques pour les entrepreneurs du secteur, avec des métriques ROI documentées et des indicateurs de performance.
\end{itemize}
\medskip
Cette approche vise à combler le gap entre la recherche académique sur l'adoption technologique et les besoins concrets des praticiens confrontés aux enjeux d'implémentation de l'IA dans leurs organisations.
