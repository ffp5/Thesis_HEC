\chapter{Revue de Littérature et Cadre Théorique}
\label{chap:literature_review}

L'adoption de l'intelligence artificielle en entreprise s'inscrit dans une longue tradition de recherches sur l'acceptation des technologies innovantes. Cette revue de littérature examine les fondements théoriques de l'adoption technologique, les spécificités de l'entrepreneurship dans ce domaine, les particularités culturelles françaises, et l'évolution du marché des services professionnels liés à la transformation digitale.

\section{Adoption Technologique et Transformation Digitale}

\subsection{Modèles Classiques d'Adoption Technologique}

Les modèles théoriques d'adoption technologique constituent le socle conceptuel pour comprendre les mécanismes d'acceptation de l'IA en entreprise. Le \textbf{Technology Acceptance Model (TAM)} de Davis \cite{davis1989perceived} reste le cadre de référence le plus utilisé dans la littérature académique. Ce modèle postule que l'intention d'utiliser une technologie dépend de deux facteurs principaux :

\begin{itemize}
    \item \textbf{L'utilité perçue} (Perceived Usefulness) : degré auquel une personne croit qu'utiliser une technologie améliorera ses performances professionnelles.
    \item \textbf{La facilité d'usage perçue} (Perceived Ease of Use) : degré auquel une personne croit que l'utilisation d'une technologie sera sans effort.
\end{itemize}

Dans le contexte de l'IA, ces variables prennent une dimension particulière. L'utilité perçue de l'IA peut être élevée (gains de productivité potentiels de 20-40\% selon nos observations), mais la facilité d'usage reste problématique en raison de la complexité perçue des technologies d'IA et du manque de formation.

La \textbf{Théorie Unifiée de l'Acceptation et de l'Utilisation de la Technologie (UTAUT)} de Venkatesh et al. \cite{venkatesh2003user} enrichit le modèle TAM en intégrant quatre déterminants clés :

\begin{enumerate}
    \item \textbf{Performance Expectancy} : attente de gains de performance.
    \item \textbf{Effort Expectancy} : effort anticipé pour maîtriser la technologie.
    \item \textbf{Social Influence} : influence de l'environnement social.
    \item \textbf{Facilitating Conditions} : conditions facilitantes organisationnelles.
\end{enumerate}

Cette approche multifactorielle s'avère particulièrement pertinente pour analyser l'adoption de l'IA en PME-ETI, où les conditions facilitantes (formation, support technique, gouvernance) jouent un rôle crucial.

\subsection{Spécificités de l'IA comme Technologie}

L'intelligence artificielle présente des caractéristiques qui la distinguent des technologies traditionnelles et complexifient son adoption. Plusieurs facteurs spécifiques émergent de la littérature récente :

\begin{itemize}
    \item \textbf{La "Black Box" et l'explicabilité} : L'IA générative, notamment, souffre d'un déficit d'explicabilité qui génère méfiance et résistance. Cette opacité contraste avec les outils informatiques traditionnels où les utilisateurs peuvent comprendre les mécanismes sous-jacents.
    \item \textbf{L'évolutivité rapide} : La vitesse d'évolution des technologies d'IA crée une anxiété technologique chez les adopteurs potentiels, qui craignent d'investir dans des solutions rapidement obsolètes.
    \item \textbf{L'ambiguïté des cas d'usage} : Contrairement aux logiciels métiers aux fonctionnalités définies, l'IA générative présente un potentiel d'application quasi infini, ce qui paradoxalement complique son adoption par manque de cas d'usage évidents.
    \item \textbf{Les enjeux éthiques et réglementaires} : L'IA Act européen et les préoccupations autour de la protection des données (RGPD) ajoutent une couche de complexité réglementaire inexistante pour d'autres technologies.
\end{itemize}

\subsection{Facteurs Organisationnels d'Adoption}

La littérature managériale identifie plusieurs facteurs organisationnels critiques pour l'adoption de l'IA :

\begin{itemize}
    \item \textbf{Le leadership et le sponsorship} : Le soutien visible de la direction générale constitue un prédicteur majeur de succès. Les études montrent que les projets IA sponsorisés au plus haut niveau ont 3,5 fois plus de chances de succès.
    \item \textbf{La culture organisationnelle} : Les entreprises dotées d'une culture d'innovation et d'expérimentation adoptent plus facilement l'IA. À l'inverse, les cultures de contrôle et de conformité génèrent des résistances.
    \item \textbf{Les compétences internes} : L'absence de compétences IA internes constitue un frein majeur, particulièrement dans les PME-ETI où les budgets formation sont contraints.
    \item \textbf{La gouvernance des données} : L'adoption de l'IA nécessite une gouvernance des données mature, prérequis souvent absent dans les organisations traditionnelles.
\end{itemize}

\section{Innovation et Entrepreneurship Technologique}

\subsection{Innovation Disruptive et IA}

La théorie de l'innovation disruptive de Christensen \cite{christensen1997innovator} offre un cadre d'analyse pertinent pour comprendre l'impact de l'IA sur les secteurs d'activité traditionnels. L'IA présente les caractéristiques d'une innovation potentiellement disruptive :

\begin{itemize}
    \item \textbf{Performance initialement inférieure} dans certains domaines (qualité des outputs, fiabilité).
    \item \textbf{Amélioration rapide} des performances techniques.
    \item \textbf{Nouvelle proposition de valeur} basée sur l'accessibilité et le coût.
    \item \textbf{Menace pour les acteurs établis} dans les services intellectuels.
\end{itemize}

Cette grille de lecture éclaire les résistances observées chez les entreprises de services traditionnelles (conseil, audit, etc.) qui voient leurs modèles économiques questionnés par l'automatisation de tâches intellectuelles.

\subsection{Dynamic Capabilities et Transformation IA}

Le concept de \textbf{Dynamic Capabilities} \cite{teece2007dynamic} s'avère particulièrement pertinent pour analyser la transformation IA des entreprises. Ces capacités dynamiques se déclinent en trois processus :

\begin{enumerate}
    \item \textbf{Sensing} : Capacité à identifier les opportunités et menaces IA.
    \item \textbf{Seizing} : Capacité à saisir ces opportunités via l'investissement et le développement.
    \item \textbf{Reconfiguring} : Capacité à reconfigurer les actifs et structures organisationnelles.
\end{enumerate}

Les PME-ETI françaises montrent souvent des lacunes dans ces trois dimensions, expliquant leur adoption lente de l'IA.

\section{Spécificités Culturelles et Organisationnelles Françaises}

\subsection{Culture Nationale et Adoption Technologique}

Les travaux de Hofstede \cite{hofstede2001culture} sur les dimensions culturelles nationales offrent un cadre d'analyse des spécificités françaises face à l'adoption technologique. Trois dimensions sont particulièrement éclairantes :

\begin{itemize}
    \item \textbf{Distance au pouvoir élevée} (68 vs 40 moyenne mondiale) : La France se caractérise par une forte hiérarchisation qui peut freiner l'adoption bottom-up de technologies comme l'IA générative, naturellement démocratisantes.
    \item \textbf{Aversion à l'incertitude forte} (86 vs 65 moyenne mondiale) : Cette caractéristique culturelle explique la préférence française pour l'encadrement réglementaire (IA Act) et la prudence face aux technologies émergentes.
    \item \textbf{Individualisme modéré} (71) : Plus faible qu'aux États-Unis (91), cette dimension favorise les approches collectives de formation et d'adoption technologique.
\end{itemize}

\subsection{PME-ETI Françaises : Caractéristiques et Enjeux}

Le tissu économique français, dominé par les PME-ETI (99,8\% des entreprises), présente des spécificités qui influencent l'adoption de l'IA \cite{bpifrance2025ia,france_strategie2025make} :

\begin{itemize}
    \item \textbf{Contraintes de ressources} : Budget et temps limités pour l'expérimentation, d'où l'importance de solutions "prêtes à l'emploi" et d'accompagnement.
    \item \textbf{Influence du dirigeant} : Dans les PME, le dirigeant-propriétaire joue un rôle déterminant dans les décisions technologiques. Sa sensibilité et ses compétences numériques conditionnent largement l'adoption.
    \item \textbf{Proximité et relations humaines} : Les PME-ETI privilégient les relations de confiance et la proximité, favorisant les prestataires locaux et l'accompagnement personnalisé.
\end{itemize}

\section{Services Professionnels et Conseil en Transformation}

\subsection{Évolution du Marché du Conseil en France}

Le marché français du conseil connaît une transformation profonde liée à la digitalisation et à l'émergence de l'IA. Plusieurs tendances structurelles se dessinent :

\begin{itemize}
    \item \textbf{Fragmentation de la demande} : Les besoins d'accompagnement IA sont plus granulaires et spécialisés que les missions de conseil traditionnelles, favorisant les boutiques spécialisées face aux grands cabinets généralistes.
    \item \textbf{Hybridation formation-conseil} : La complexité de l'IA génère une demande forte de montée en compétences couplée aux missions de conseil, créant de nouveaux modèles hybrides.
\end{itemize}

\subsection{Business Models Émergents}

L'accompagnement à l'IA génère l'émergence de nouveaux modèles d'affaires hybrides :

\begin{itemize}
    \item \textbf{Formation + Conseil + Delivery} : Modèle intégré proposant sensibilisation, cadrage stratégique et implémentation opérationnelle. C'est le positionnement adopté par Luwai après plusieurs itérations.
    \item \textbf{SaaS + Services} : Couplage d'une plateforme technologique avec des services d'accompagnement, modèle adopté par de nombreuses startups IA B2B.
\end{itemize}

\section{Synthèse du Cadre Théorique}

Cette revue de littérature révèle plusieurs gaps théoriques et pratiques que cette recherche ambitionne de combler :

\begin{itemize}
    \item \textbf{Gap empirique} : Peu d'études qualitatives approfondies sur l'adoption de l'IA dans les PME-ETI françaises, segment pourtant critique pour l'économie nationale.
    \item \textbf{Gap théorique} : Les modèles d'adoption technologique classiques (TAM, UTAUT) nécessitent une adaptation au contexte spécifique de l'IA et aux particularités culturelles françaises.
    \item \textbf{Gap pratique} : Manque de frameworks opérationnels pour guider les entrepreneurs dans la construction de modèles d'affaires viables sur le marché de l'accompagnement IA.
\end{itemize}

Le cas Luwai, analysé dans la partie suivante, permet d'explorer ces gaps à travers l'expérience concrète d'un entrepreneur confronté aux réalités du terrain.
