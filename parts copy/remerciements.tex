Cette thèse n'aurait pas pu voir le jour sans le soutien et la contribution de nombreuses personnes que je tiens à remercier chaleureusement.
\\\\
\textbf{À l'équipe Luwai,} mes cofondateurs et collaborateurs qui partagent cette vision de démocratisation de l'IA en France. Merci en particulier à Miguel Adolf Torres pour son expertise technique complémentaire et sa vision stratégique.
\\\\
\textbf{Aux 63 prospects et clients} qui ont accepté de partager leur temps et leur vision lors des entretiens qui constituent le cœur empirique de cette recherche. Votre franchise et votre ouverture ont permis de dresser un diagnostic précis des enjeux d'adoption de l'IA dans le tissu économique français.
\\\\
\textbf{À la communauté MS X-HEC Entrepreneurs,} promotions 2024 et 2025, pour les échanges enrichissants et les remises en question constructives tout au long de ce parcours. L'émulation collective a été un moteur essentiel de cette réflexion.
\\\\
\textbf{À nos premiers clients et partenaires} - Aesio, Antilogy, Intégrhale, Carecall, Tectona - qui nous ont fait confiance pour les accompagner dans leur transformation et ont validé par leurs résultats la pertinence de notre approche.
\\\\
\textbf{Aux mentors et advisors} qui ont contribué à affiner la vision stratégique de Luwai et ont nourri les réflexions présentées dans cette thèse.
\\\\
\textbf{À l'École Polytechnique et HEC Paris} pour la richesse de leurs enseignements et l'ouverture sur l'écosystème entrepreneurial qui ont rendu possible ce projet.
\\\\
Cette thèse est le fruit d'un apprentissage collectif autant qu'individuel. Elle témoigne de la puissance de l'écosystème français d'innovation quand il sait combiner excellence académique et pragmatisme entrepreneurial.
