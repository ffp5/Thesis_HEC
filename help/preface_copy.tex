\begin{quote}
\textit{"Il y a deux types d'entreprises : celles qui s'adaptent à l'IA et celles qui disparaissent."} \\
--- Jensen Huang, CEO de NVIDIA
\end{quote}
\medskip

L’intelligence artificielle est aujourd’hui l’une des transformations technologiques les plus profondes de notre époque. Pourtant, en France, cette révolution semble avancer à deux vitesses : d’un côté, un \textbf{écosystème startup dynamique} et des \textbf{avancées réglementaires pionnières} avec l’IA Act européen ; de l’autre, des \textbf{PME-ETI} qui peinent à concrétiser le potentiel de ces technologies dans leur quotidien opérationnel.
\\\\
Cette thèse prend racine dans un \textbf{choc culturel personnel} vécu lors d’un échange de trois mois à San Francisco. En tant qu’ingénieur français immergé dans l’écosystème de la Silicon Valley, j’ai été témoin d’une adoption massive et naturelle de l’IA dans tous les secteurs. \textbf{Appels d’offres automatisés}, \textbf{due diligences accélérées}, \textbf{créations de contenu optimisées} : l’IA était omniprésente, non pas comme une technologie futuriste, mais comme un outil de productivité aussi banal qu’Excel.
\\\\
Le contraste a été saisissant à mon retour en France. Malgré notre \textbf{excellence technologique} et notre \textbf{écosystème d’innovation reconnu}, j’ai découvert un \textbf{gap considérable} entre le \textbf{potentiel théorique} de l’IA et son \textbf{adoption effective} dans le tissu économique français. Cette observation m’a conduit à créer \textbf{Luwai} en 2025, avec pour mission de transformer les entreprises françaises d’\textit{“AI-curious”} à \textit{“AI-productive”}.
\\\\
Cette thèse relate cette \textbf{aventure entrepreneuriale} tout en questionnant profondément les raisons des réticences vis-à-vis de l’IA en France. Elle mobilise \textbf{500 relances téléphoniques (call-backs)}, \textbf{63 entretiens semi-directifs} menés auprès de prospects et de clients, \textbf{5 propositions commerciales} rédigées par moi-même et l’expérience vécue de la réalisation d’un \textbf{business model} dans un secteur d’activité en forte expansion.
\\\\
Le parti pris est d’être résolument celui d’un \textbf{entrepreneur} : comment créer de la valeur à partir des freins identifiés ? Comment rallier les \textbf{innovations de rupture} des ingénieurs à la \textbf{réalité terrain} des managers français ?
\\\\
Si cette thèse est nourrie par cette grande enquête, elle se veut aussi une forme de \textbf{guide des bonnes pratiques} à destination des entrepreneurs tentés par l’aventure du coaching par l’IA et un \textbf{outil de questionnement} des dirigeants de PME-ETI face à ces enjeux.